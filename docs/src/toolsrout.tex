\section{Data management routines}
\label{sec:toolsrout}


%
%%   psb_cdall %%
%
\subsection{psb\_cdall --- Allocates a communication descriptor}

\begin{verbatim}
call psb_cdall(icontxt, desc_a, info,mg=mg,parts=parts)
call psb_cdall(icontxt, desc_a, info,vg=vg,[mg=mg,flag=flag])
call psb_cdall(icontxt, desc_a, info,vl=vl,[nl=nl,globalcheck=.false.,lidx=lidx])
call psb_cdall(icontxt, desc_a, info,nl=nl)
call psb_cdall(icontxt, desc_a, info,mg=mg,repl=.true.)
\end{verbatim}

This subroutine initializes the communication descriptor associated
with an index space. One of the  optional arguments
\verb|parts|, \verb|vg|, \verb|vl|, \verb|nl|  or \verb|repl| 
must be specified, thereby choosing
the specific initialization strategy.
\begin{description}
\item[\bf  On Entry ]
\item[Type:] Synchronous.
\item[icontxt] the communication context.\\
Scope:{\bf global}.\\
Type:{\bf required}.\\
Intent: {\bf in}.\\
Specified as: an integer value.
\item[vg] Data allocation: each index $i\in \{1\dots mg\}$ is allocated
  to process $vg(i)$.\\
Scope:{\bf global}.\\
Type:{\bf optional}.\\
Intent: {\bf in}.\\
Specified as: an integer array. 
\item[flag] Specifies whether entries in $vg$ are zero- or one-based.\\
Scope:{\bf global}.\\
Type:{\bf optional}.\\
Intent: {\bf in}.\\
Specified as: an integer value $0,1$, default $0$.

\item[mg] the (global) number of rows of the problem.\\
Scope:{\bf global}.\\
Type:{\bf optional}.\\
Intent: {\bf in}.\\
Specified as: an integer value. It is required if \verb|parts| or
\verb|repl| is specified, it is optional if \verb|vg| is specified. 
\item[parts] the subroutine that defines the partitioning scheme.\\
Scope:{\bf global}.\\
Type:{\bf required}.\\
Specified as: a subroutine. 
\item[vl] Data allocation: the set of global indices 
  $vl(1:nl)$ belonging to the calling process. \\
Scope:{\bf local}.\\
Type:{\bf optional}.\\
Intent: {\bf in}.\\
Specified as: an integer array. 
\item[nl] Data allocation: in a generalized block-row distribution the
  number of indices belonging to the current process. \\
Scope:{\bf local}.\\
Type:{\bf optional}.\\
Intent: {\bf in}.\\
Specified as: an integer value. May be specified together with
\verb|vl|. 
\item[repl] Data allocation: build a replicated index space 
  (i.e. all processes own all indices).\\
Scope:{\bf global}.\\
Type:{\bf optional}.\\
Intent: {\bf in}.\\
Specified as: the logical value \verb|.true.|
\item[globalcheck] Data allocation: do global checks on the local
  index lists \verb|vl|\\
Scope:{\bf global}.\\
Type:{\bf optional}.\\
Intent: {\bf in}.\\
Specified as: a logical value, default: \verb|.false.|
\item[lidx] Data allocation: the set of local  indices 
  $lidx(1:nl)$ to be assigned to the global indices $vl$. \\
Scope:{\bf local}.\\
Type:{\bf optional}.\\
Intent: {\bf in}.\\
Specified as: an integer array. 
\end{description}

\begin{description}
\item[\bf On Return]
\item[desc\_a] the communication descriptor.\\
Scope:{\bf local}.\\
Type:{\bf required}.\\
Intent: {\bf out}.\\
Specified as: a structured data of type \descdata.
\item[info] Error code.\\
Scope: {\bf local} \\
Type: {\bf required} \\
Intent: {\bf out}.\\
An integer value; 0 means no error has been detected. 
\end{description}

{\par\noindent\large\bfseries Notes}
\begin{enumerate}
\item One of the optional arguments  \verb|parts|, \verb|vg|,
  \verb|vl|, \verb|nl|  or \verb|repl| must be specified, thereby choosing the
  initialization   strategy as follows:
\begin{description}
\item[parts] In this case we have a subroutine specifying the mapping
  between global indices and process/local index pairs. If this
  optional argument   is specified, then it is mandatory to 
  specify the argument \verb|mg| as well.  
  The subroutine must conform to the following interface: 
\begin{verbatim}
  interface 
     subroutine psb_parts(glob_index,mg,np,pv,nv)
       integer, intent (in)  :: glob_index,np,mg
       integer, intent (out) :: nv, pv(*)
     end subroutine psb_parts
  end interface
\end{verbatim}
  The input arguments are:
  \begin{description}
    \item[glob\_index] The global index to be mapped;
    \item[np] The number of processes in the mapping;
    \item[mg] The total number of global rows in the mapping;
  \end{description}
  The output arguments are:
  \begin{description}
    \item[nv] The number of entries in \verb|pv|;
    \item[pv] A vector containing the indices of the processes to
    which the global index should be assigend; each entry must satisfy
    $0\le pv(i) < np$; if $nv>1$ we have an index assigned to multiple
    processes, i.e. we have an overlap among the subdomains.
  \end{description}
\item[vg] In this case the association between an index and a process
  is specified via an integer vector  \verb|vg(1:mg)|; 
  each index $i \in \{1\dots mg\}$  is assigned to process $vg(i)$. 
  The vector \verb|vg| must be identical on all
  calling processes; its entries may have the ranges $(0\dots np-1)$
  or $(1\dots np)$ according to the value of \verb|flag|.  
  The size $mg$ may be  specified via the optional argument \verb|mg|;
  the default is to use the entire vector \verb|vg|, thus having
  \verb|mg=size(vg)|. 
\item[vl] In this case we are specifying the list of indices
  \verb|vl(1:nl)| assigned   to the current process; thus, the global
  problem size $mg$ is given by 
  the range of the aggregate of the individual vectors \verb|vl| specified
  in the calling processes. The size may be specified via the optional
  argument \verb|nl|; the default is to use the entire vector
  \verb|vl|, thus having   \verb|nl=size(vl)|.  
  If \verb|globalcheck=.true.|   the subroutine will check how many
  times  each entry in the global   index   space $(1\dots mg)$ is
  specified in the input lists   \verb|vl|, thus allowing for the
  presence of overlap in the input,   and checking for ``orphan''
  indices.    If \verb|globalcheck=.false.|, the subroutine will not
  check for   overlap, and may be significantly faster, but   the user
  is implicitly guaranteeing that there are neither orphan   nor
  overlap indices.  
\item[lidx] The optional argument \verb|lidx| is available for
  those cases in which the user has already established a
  global-to-local mapping; if it is specified, each  index in
  \verb|vl(i)| will be mapped to the corresponding local index
  \verb|lidx(i)|. When specifying the argument \verb|lidx| the user
  would also likely employ \verb|lidx| in calls to \verb|psb_cdins|
  and \verb|local| in calls to \verb|psb_spins| and \verb|psb_geins|;
  see also sec.~\ref{sec:usermaps}. 
\item[nl] If this argument is specified alone (i.e. without \verb|vl|)
  the  result is  a generalized row-block   distribution in which each
  process $I$ gets assigned a consecutive   chunk of $N_I=nl$ global
  indices. 
\item[repl] This arguments specifies to replicate all indices on
  all processes. This is a special purpose data allocation that is
  useful in the construction of some multilevel preconditioners.
\end{description}
\item On exit from this routine the descriptor is in the build
  state. 
\item Calling the routine with \verb|vg| or \verb|parts| implies that
  every process will scan the entire index space to figure out the
  local indices.
\item Overlapped indices are possible with both \verb|parts| and
  \verb|vl| invocations.
\item When the subroutine is invoked with \verb|vl| in
  conjunction with \verb|globalcheck=.true.|, it will perform a scan
  of the index   space to search for overlap or orphan indices.
\item When the subroutine is invoked with \verb|vl| in
  conjunction with \verb|globalcheck=.false.|, no index space scan
  will take place. Thus it is the responsibility of the user to make
  sure that the indices specified in \verb|vl| have neither orphans nor
  overlaps; if this assumption fails, results will be 
  unpredictable. 
\item Orphan and overlap indices are
  impossible by construction when the subroutine is invoked with
  \verb|nl| (alone),  or \verb|vg|.
\end{enumerate}


%
%%   psb_cdins %%
%
\clearpage\subsection{psb\_cdins --- Communication descriptor insert
  routine}

\begin{verbatim}
call psb_cdins(nz, ia, ja, desc_a, info [,ila,jla])
call psb_cdins(nz,ja,desc,info[,jla,mask,lidx])
\end{verbatim}

This subroutine examines the edges of the graph associated with the
discretization mesh (and isomorphic to the sparsity pattern of a
linear system coefficient matrix), storing them as necessary into the
communication descriptor. In the first form the edges are specified as
pairs of indices $ia(i),ja(i)$; the starting index $ia(i)$ should
belong to the current process. 
In the second form only the remote indices $ja(i)$ are specified. 

\begin{description}
\item[Type:] Asynchronous.
\item[\bf On Entry]
\item[nz] the number of points being inserted.\\
Scope: {\bf local}.\\
Type: {\bf required}.\\
Intent: {\bf in}.\\
Specified as: an integer value.
\item[ia] the indices of the starting vertex of the edges  being inserted.\\
Scope: {\bf local}.\\
Type: {\bf required}.\\
Intent: {\bf in}.\\
Specified as: an integer array of length $nz$.
\item[ja]  the indices of the end vertex of the edges  being inserted.\\
Scope: {\bf local}.\\
Type: {\bf required}.\\
Intent: {\bf in}.\\
Specified as: an integer array of length $nz$.
\item[mask] Mask entries in \verb|ja|, they are inserted only when the
  corresponding \verb|mask| entries are \verb|.true.|\\
Scope: {\bf local}.\\
Type: {\bf optional}.\\
Intent: {\bf in}.\\
Specified as: a logical  array of length $nz$, default \verb|.true.|.
\item[lidx]  User defined local indices for \verb|ja|.\\
Scope: {\bf local}.\\
Type: {\bf optional}.\\
Intent: {\bf in}.\\
Specified as: an integer array of length $nz$.
%% \item[is] the row offset.\\
%% Scope:{\bf local}.\\
%% Type:{\bf optional}.\\a
%% Specified as: an integer value.
%% \item[js] the column offset.\\
%% Scope: {\bf local}.\\
%% Type: {\bf optional}.\\
%% Specified as: an integer value.
\end{description}

\begin{description}
\item[\bf On Return]
\item[desc\_a] the updated communication descriptor.\\
Scope:{\bf local}.\\
Type:{\bf required}.\\
Intent: {\bf inout}.\\
Specified as: a structured data of type \descdata.
\item[info] Error code.\\
Scope: {\bf local} \\
Type: {\bf required} \\
Intent: {\bf out}.\\
An integer value; 0 means no error has been detected. 
\item[ila] the local indices of the starting vertex of the edges  being inserted.\\
Scope: {\bf local}.\\
Type: {\bf optional}.\\
Intent: {\bf out}.\\
Specified as: an integer array of length $nz$.
\item[jla]  the local indices of the end vertex of the edges  being inserted.\\
Scope: {\bf local}.\\
Type: {\bf optional}.\\
Intent: {\bf out}.\\
Specified as: an integer array of length $nz$.

\end{description}
{\par\noindent\large\bfseries Notes}
\begin{enumerate}
\item This routine may only be called if the descriptor is in the
  build state;
\item  This routine automatically ignores edges that do not
insist on the  current process, i.e. edges for which neither the starting
nor the end vertex belong to the current process.
\item The second form of this routine will be useful when dealing with
  user-specified index mappings; see also~\ref{sec:usermaps}.
\end{enumerate}



%
%%   psb_cdasb %%
%
\clearpage\subsection{psb\_cdasb --- Communication descriptor assembly
  routine}

\begin{verbatim}
call psb_cdasb(desc_a, info [, mold])
\end{verbatim}

\begin{description}
\item[Type:] Synchronous.
\item[\bf On Entry]
\item[desc\_a] the communication descriptor.\\
Scope:{\bf local}.\\
Type:{\bf required}.\\
Intent: {\bf inout}.\\
Specified as: a structured data of type \descdata.
\item[mold] The desired dynamic type for the internal index storage.\\
Scope: {\bf local}.\\
Type: {\bf optional}.\\
Intent: {\bf in}.\\
Specified as: a object of type derived from (integer) \vbasedata.
\end{description}

\begin{description}
\item[\bf On Return]
\item[desc\_a] the communication descriptor.\\
Scope:{\bf local}.\\
Type:{\bf required}.\\
Intent: {\bf inout}.\\
Specified as: a structured data of type \descdata.
\item[info] Error code.\\
Scope: {\bf local} \\
Type: {\bf required} \\
Intent: {\bf out}.\\
An integer value; 0 means no error has been detected. 
%\item[arg] 
\end{description}
{\par\noindent\large\bfseries Notes}
\begin{enumerate}
\item On exit from this routine the descriptor is in the assembled
  state. 
\end{enumerate}
This call will set up all the necessary information for the halo data
exchanges. In doing so, the library will need to identify the set of
processes owning the halo indices through the use of the
\verb|desc%fnd_owner()|  method; the owning processes are the
topological neighbours of the calling process. If the user has some
background information on the processes that are neighbours of the
current one, it is possible to specify explicitly the list of adjacent
processes with a call to  \verb|desc%set_p_adjcncy(list)|; this will
speed up the subsequent call to  \verb|psb_cdasb|. 

%
%%   psb_cdcpy %%
%
\clearpage\subsection{psb\_cdcpy --- Copies a communication descriptor}

\begin{verbatim}
call psb_cdcpy(desc_in, desc_out, info)
\end{verbatim}

\begin{description}
\item[Type:] Asynchronous.
\item[\bf On Entry]
\item[desc\_in] the communication descriptor.\\
Scope:{\bf local}.\\
Type:{\bf required}.\\
Intent: {\bf in}.\\
Specified as: a structured data of type \descdata.

\end{description}

\begin{description}
\item[\bf On Return]
\item[desc\_out] the communication descriptor copy.\\
Scope:{\bf local}.\\
Type:{\bf required}.\\
Intent: {\bf out}.\\
Specified as: a structured data of type \descdata.
\item[info] Error code.\\
Scope: {\bf local} \\
Type: {\bf required} \\
Intent: {\bf out}.\\
An integer value; 0 means no error has been detected. 
\end{description}


%
%%   psb_cdfree %%
%
\clearpage\subsection{psb\_cdfree --- Frees a communication descriptor}

\begin{verbatim}
call psb_cdfree(desc_a, info)
\end{verbatim}

\begin{description}
\item[Type:] Synchronous.
\item[\bf On Entry]
\item[desc\_a] the communication descriptor to be freed.\\
Scope:{\bf local}.\\
Type:{\bf required}.\\
Intent: {\bf inout}.\\
Specified as: a structured data of type \descdata.
\end{description}

\begin{description}
\item[\bf On Return]
\item[info] Error code.\\
Scope: {\bf local} \\
Type: {\bf required} \\
Intent: {\bf out}.\\
An integer value; 0 means no error has been detected. 
\end{description}



%
%%   psb_cdcpy %%
%
\clearpage\subsection{psb\_cdbldext --- Build an extended  communication
  descriptor}

\begin{verbatim}
call psb_cdbldext(a,desc_a,nl,desc_out, info, extype)
\end{verbatim}

This subroutine builds an extended communication descriptor, based on
the input descriptor \verb|desc_a| and on the stencil specified
through the input sparse matrix \verb|a|. 
\begin{description}
\item[Type:] Synchronous.
\item[\bf On Entry]
\item[a] A sparse matrix
Scope:{\bf local}.\\
Type:{\bf required}.\\
Intent: {\bf in}.\\
Specified as: a structured data type.
\item[desc\_a] the communication descriptor.\\
Scope:{\bf local}.\\
Type:{\bf required}.\\
Intent: {\bf in}.\\
Specified as: a structured data of type \spdata.
\item[nl] the number of additional layers desired.\\
Scope:{\bf global}.\\
Type:{\bf required}.\\
Intent: {\bf in}.\\
Specified as: an integer value $nl\ge 0$. 
\item[extype] the kind of estension required.\\
Scope:{\bf global}.\\
Type:{\bf optional }.\\
Intent: {\bf in}.\\
Specified as: an integer value
\verb|psb_ovt_xhal_|, \verb|psb_ovt_asov_|, default: \verb|psb_ovt_xhal_|

\end{description}

\begin{description}
\item[\bf On Return]
\item[desc\_out] the extended communication descriptor.\\
Scope:{\bf local}.\\
Type:{\bf required}.\\
Intent: {\bf inout}.\\
Specified as: a structured data of type \descdata.
\item[info] Error code.\\
Scope: {\bf local} \\
Type: {\bf required} \\
Intent: {\bf out}.\\
An integer value; 0 means no error has been detected. 
\end{description}

{\par\noindent\large\bfseries Notes}
\begin{enumerate}
\item Specifying \verb|psb_ovt_xhal_| for the \verb|extype| argument
  the user will obtain a descriptor for a domain partition in which
  the additional layers are fetched as part of an (extended) halo;
  however the index-to-process mapping is identical to that of the
  base descriptor;
\item Specifying \verb|psb_ovt_asov_| for the \verb|extype| argument
  the user will obtain a descriptor with an overlapped decomposition:
  the additional layer is aggregated to  the local subdomain (and thus
  is an overlap), and a new halo extending beyond the last additional
  layer is formed. 
\end{enumerate}


%% %
%% %%   psb_cdren %%
%% %
%% \subsection{psb\_cdren --- Applies a renumeration to a
%% communication descriptor}
%%% \addcontentsline{toc}{subsection}{psb\_cdren}

%% \syntax{call psb\_cdren}{trans, iperm, desc\_a, info}

%% \begin{description}
%% \item[\bf On Entry]
%% \item[Type:] Asynchronous.
%% \item[trans] A character that specifies whether to permute $A$  or $A^T$.\\
%% Scope: {\bf local} \\
%% Type: {\bf required}\\
%% Specified as: a single character with value 'N' for $A$ or 'T' for $A^T$.\\
%% \item[iperm] An integer array containing permutation information.\\
%% Scope: {\bf local} \\
%% Type: {\bf required}\\
%% Specified as: an integer one-dimensional array.\\
%% \item[desc\_a] the communication descriptor.\\
%% Scope:{\bf local}.\\
%% Type:{\bf required}.\\
%% Specified as: a structured data of type \descdata.
%% \end{description}

%% \begin{description}
%% \item[\bf On Return]
%% \item[info] Error code.
%% Scope: {\bf local} \\
%% Type: {\bf required}\\
%% Specified as: an integer variable.
%% \end{description}




%
%%   psb_descprt %%
%
%% \subsection{psb\_cdprt --- Prints a descriptor}

%% \syntax{call psb\_cdprt}{iout, desc\_a, glob, short}

%% \begin{description}
%% \item[Type:] Asynchronous.
%% \item[\bf On Entry]
%% \item[iout] An integer that defines the output unit.
%% Scope: {\bf local} \\
%% Type: {\bf required}\\
%% Specified as: Integer scalar.\\
%% \item[desc\_a] The communication descriptor of type \descdata that
%%   must be printed.\\
%% Scope: {\bf local} \\
%% Type: {\bf required}\\
%% Specified as: a variable of type \descdata.\\
%% \end{description}

%% \begin{description}
%% \item[\bf On Return]
%% \item[glob] ??????
%% \item[short] ??????
%% \end{description}



%
%%   psb_spalloc %%
%
\clearpage\subsection{psb\_spall --- Allocates a sparse matrix}

\begin{verbatim}
call psb_spall(a, desc_a, info [, nnz, dupl, buildmode])
\end{verbatim}

\begin{description}
\item[Type:] Synchronous.
\item[\bf On Entry]
\item[desc\_a] the communication descriptor.\\
Scope:{\bf local}.\\
Type:{\bf required}.\\
Intent: {\bf in}.\\
Specified as: a structured data of type \descdata.
\item[nnz] An estimate of the number of nonzeroes in the local
  part of the assembled matrix.\\ 
Scope: {\bf global}.\\
Type: {\bf optional}.\\
Intent: {\bf in}.\\
Specified as: an integer value. 
\item[dupl] How to handle duplicate coefficients.\\
Scope: {\bf global}.\\
Type: {\bf optional}.\\
Intent: {\bf in}.\\
Specified as: integer, possible values: \verb|psb_dupl_ovwrt_|,
\verb|psb_dupl_add_|, \verb|psb_dupl_err_|.
\item[buildmode] Whether to keep track of matrix entries that do not
  belong to the current process.\\ 
Scope: {\bf global}.\\
Type: {\bf optional}.\\
Intent: {\bf in}.\\
Specified as: an integer value \verb|psb_matbld_noremote_|,
\verb|psb_matbld_remote_|.  Default:  \verb|psb_matbld_noremote_|.
\end{description}

\begin{description}
\item[\bf On Return]
\item[a] the matrix to be allocated.\\
Scope:{\bf local}\\
Type:{\bf required}\\
Intent: {\bf out}.\\
Specified as: a structured data of type \spdata.
\item[info] Error code.\\
Scope: {\bf local} \\
Type: {\bf required} \\
Intent: {\bf out}.\\
An integer value; 0 means no error has been detected. 
\end{description}
{\par\noindent\large\bfseries Notes}
\begin{enumerate}
\item On exit from this routine the sparse matrix  is in the build
  state.
\item The descriptor may be in either the build or assembled state.
\item Providing a good estimate for the number of nonzeroes $nnz$ in
  the assembled matrix may substantially improve performance in the
  matrix build phase, as it will reduce or eliminate the need for
  (potentially multiple) data reallocations;
\item Using \verb|psb_matbld_remote_| will cause an overhead at
  assembly time;
\end{enumerate}



%
%%   psb_spins %%
%
\clearpage\subsection{psb\_spins --- Insert a set of coefficients into a sparse
  matrix}

\begin{verbatim}
call psb_spins(nz, ia, ja, val, a, desc_a, info [,local])
call psb_spins(nr, irw, irp, ja, val, a, desc_a, info [,local])
\end{verbatim}

\begin{description}
\item[Type:] Asynchronous.
\item[\bf On Entry]
\item[nz] the number of coefficients to be inserted.\\
Scope:{\bf local}.\\
Type:{\bf required}.\\
Intent: {\bf in}.\\
Specified as: an integer scalar.
\item[nr] the number of rows to be inserted.\\
Scope:{\bf local}.\\
Type:{\bf required}.\\
Intent: {\bf in}.\\
Specified as: an integer scalar.
\item[irw] the first  row to be inserted.\\
Scope:{\bf local}.\\
Type:{\bf required}.\\
Intent: {\bf in}.\\
Specified as: an integer scalar.
\item[ia] the row indices of the coefficients to be inserted.\\
Scope:{\bf local}.\\
Type:{\bf required}.\\
Intent: {\bf in}.\\
Specified as: an integer array of size $nz$.
\item[irp] the row pointers of the coefficients to be inserted.\\
Scope:{\bf local}.\\
Type:{\bf required}.\\
Intent: {\bf in}.\\
Specified as: an integer array of size $nr+1$.
\item[ja] the column indices of the coefficients to be inserted.\\
Scope:{\bf local}.\\
Type:{\bf required}.\\
Intent: {\bf in}.\\
Specified as: an integer array of size $nz$.
\item[val] the coefficients to be inserted.\\
Scope:{\bf local}.\\
Type:{\bf required}.\\
Intent: {\bf in}.\\
Specified as: an array of size $nz$. Must be of the same type and kind
of the coefficients  of the sparse matrix $a$.
\item[desc\_a] The communication descriptor.\\
Scope: {\bf local}. \\
Type: {\bf required}.\\
Intent: {\bf inout}.\\
Specified as: a variable of type \descdata.\\
\item[local] Whether the entries in the indices vectors \verb|ia|,
  \verb|ja| are already in local  numbering. \\
 Scope:{\bf local}.\\
 Type:{\bf optional}.\\
 Specified as: a logical value; default: \verb|.false.|.

%% \item[is] the starting row on matrix $a$.\\
%% Scope:{\bf local}.\\
%% Type:{\bf optional}.\\
%% Specified as: an integer vaule.
%% \item[js] the starting column on matrix $a$.\\
%% Scope:{\bf local}.\\
%% Type:{\bf optional}\\
%% Specified as: an integer value
\end{description}

\begin{description}
\item[\bf On Return]
\item[a] the matrix into which coefficients will be inserted.\\
Scope:{\bf local}\\
Type:{\bf required}\\
Intent: {\bf inout}.\\
Specified as: a structured data of type \spdata.
\item[desc\_a] The communication descriptor.\\
Scope: {\bf local}. \\
Type: {\bf required}.\\
Intent: {\bf inout}.\\
Specified as: a variable of type \descdata.\\
\item[info] Error code.\\
Scope: {\bf local} \\
Type: {\bf required} \\
Intent: {\bf out}.\\
An integer value; 0 means no error has been detected. 
\end{description}

{\par\noindent\large\bfseries Notes}
\begin{enumerate}
\item On entry to this routine the descriptor may be in either the
  build or assembled state.
\item On entry to this routine the sparse matrix may be in either the
  build or update state. 
\item If the descriptor is in the build state, then the sparse matrix
  must also be in the build state; the action of the routine is to
  (implicitly) call \verb|psb_cdins| to add entries to the sparsity
  pattern; each sparse matrix entry implicitly defines a graph edge,
  that is passed to the descriptor routine for the appropriate
  processing;
\item The input data can be passed in either COO or CSR formats;
\item In COO format the coefficients to be inserted are represented by
  the ordered   triples $ia(i),ja(i),val(i)$, for $i=1,\dots,nz$;
  these triples  should belong to the current process, i.e. $ia(i)$
  should be one of  the local indices, but are otherwise arbitrary;
\item In CSR format the coefficients to be inserted for each input row $i=1,nr$
  are represented by the ordered triples $(i+irw-1),ja(j),val(j)$, for
  $j=irp(i),\dots,irp(i+1)-1$; 
  these triples  should belong to the current process, i.e. $i+irw-1$
  should be one of  the local indices, but are otherwise arbitrary;  
\item There is no   requirement that a given row must be passed in its
  entirety to a   single call to this routine: the buildup of a row
  may be split into   as many calls as desired (even in the CSR format); 
\item Coefficients from different rows may also be mixed up freely
  in a single call, according to the application needs; 
\item Any coefficients from matrix rows not owned by the calling
  process are silently ignored;
\item If the descriptor is in the assembled state, then any entries in
  the sparse matrix that would generate additional communication
  requirements are  ignored; 
\item If the matrix is in the update state, any entries in positions
  that were not present in the original matrix are ignored. 
\end{enumerate}

%
%%   psb_spasb %%
%
\clearpage\subsection{psb\_spasb --- Sparse matrix assembly routine}

\begin{verbatim}
call psb_spasb(a, desc_a, info, afmt, upd,  mold)
\end{verbatim}

\begin{description}
\item[Type:] Synchronous.
\item[\bf On Entry]
\item[desc\_a] the communication descriptor.\\
Scope:{\bf local}.\\
Type:{\bf required}.\\
Intent: {\bf in/out}.\\
Specified as: a structured data of type \descdata.
\item[afmt] the storage format for the sparse matrix.\\
Scope: {\bf local}.\\
Type: {\bf optional}.\\
Intent: {\bf in}.\\
Specified as: an array of characters. Defalt:  'CSR'.
\item[upd] Provide for updates to the matrix coefficients.\\
Scope: {\bf global}.\\
Type: {\bf optional}.\\
Intent: {\bf in}.\\
Specified as: integer, possible values: \verb|psb_upd_srch_|, \verb|psb_upd_perm_|
\item[mold] The desired dynamic type for the internal matrix storage.\\
Scope: {\bf local}.\\
Type: {\bf optional}.\\
Intent: {\bf in}.\\
Specified as: an object of a class derived from \spbasedata. 
\end{description}

\begin{description}
\item[\bf On Return]
\item[a] the matrix to be assembled.\\
Scope:{\bf local}\\
Type:{\bf required}\\
Intent: {\bf inout}.\\
Specified as: a structured data of type \spdata.
\item[desc\_a] the communication descriptor.\\
Scope:{\bf local}.\\
Type:{\bf required}.\\
Intent: {\bf in/out}.\\
Specified as: a structured data of type \descdata. If the matrix was
allocated with \verb|buildmode=psb_matbld_remote_|, then the
descriptor will be reassembled. 
\item[info] Error code.\\
Scope: {\bf local} \\
Type: {\bf required} \\
Intent: {\bf out}.\\
An integer value; 0 means no error has been detected. 
\end{description}

{\par\noindent\large\bfseries Notes}
\begin{enumerate}
\item On entry to this routine the descriptor must  be in  the
  assembled state, i.e. \verb|psb_cdasb| must already have been called.
\item The sparse matrix may be in either the build or update state;
\item Duplicate entries are detected and handled in both build and
  update state, with the exception of the error action that is only
  taken in the build state, i.e. on the first assembly; 
\item If the update choice is \verb|psb_upd_perm_|, then subsequent
  calls to \verb|psb_spins| to update the matrix must be arranged in
  such a way as to produce exactly the same sequence of coefficient
  values as encountered at the first assembly; 
\item The output storage format need not be the same on all
  processes; 
\item On exit from this routine the matrix is in the assembled state,
  and thus is suitable for the computational routines;
\item If the \verb|buildmode=psb_matbld_remote_| value was specified
  at allocation time, contributions defined on the current process but
  belonging to a remote process will be handled accordingly. This is
  most likely to occur in finite element applications, with
  \verb|dupl=psb_dupl_add_|; note that the descriptor will be updated,
  and there will be a runtime overhead. 
\end{enumerate}



%% %
%% %%   psb_spcnv %%
%% %
%% \subsection{psb\_spcnv --- Converts a sparse matrix storage
%% format}

%% \syntax{call psb\_spcnv}{a, b, desc\_a, info}

%% \begin{description}
%% \item[\bf On Entry]
%% \item[a] the matrix to be converted.\\
%% Scope:{\bf local}\\
%% Type:{\bf required}\\
%% Specified as: a structured data of type \spdata.
%% \item[desc\_a] the communication descriptor.\\
%% Scope:{\bf local}.\\
%% Type:{\bf required}.\\
%% Specified as: a structured data of type \descdata.
%% \end{description}

%% \begin{description}
%% \item[\bf On Return]
%% \item[b] the converted matrix.\\
%% Scope:{\bf local}\\
%% Type:{\bf required}\\
%% Specified as: a structured data of type \spdata.
%% \item[info] Error code.
%% Scope: {\bf local} \\
%% Type: {\bf required}\\
%% Specified as: an integer variable.
%% \end{description}



%
%%   psb_spfree %%
%
\clearpage\subsection{psb\_spfree --- Frees a sparse matrix}

\begin{verbatim}
call psb_spfree(a, desc_a, info)
\end{verbatim}

\begin{description}
\item[Type:] Synchronous.
\item[\bf On Entry]
\item[a] the matrix to be freed.\\
Scope:{\bf local}\\
Type:{\bf required}\\
Intent: {\bf inout}.\\
Specified as: a structured data of type \spdata.
\item[desc\_a] the communication descriptor.\\
Scope:{\bf local}.\\
Type:{\bf required}.\\
Intent: {\bf in}.\\
Specified as: a structured data of type \descdata.
\end{description}

\begin{description}
\item[\bf On Return]
\item[info] Error code.\\
Scope: {\bf local} \\
Type: {\bf required} \\
Intent: {\bf out}.\\
An integer value; 0 means no error has been detected. 
\end{description}




%
%%   psb_sprn %%
%
\clearpage\subsection{psb\_sprn --- Reinit sparse matrix structure for psblas
  routines.}

\begin{verbatim}
call psb_sprn(a, decsc_a, info, clear)
\end{verbatim}

\begin{description}
\item[Type:] Synchronous.
\item[\bf On Entry]
\item[a] the matrix to be reinitialized.\\
Scope:{\bf local}\\
Type:{\bf required}\\
Intent: {\bf inout}.\\
Specified as: a structured data of type \spdata.
\item[desc\_a] the communication descriptor.\\
Scope:{\bf local}.\\
Type:{\bf required}.\\
Intent: {\bf in}.\\
Specified as: a structured data of type \descdata.
\item[clear] Choose whether to zero out matrix coefficients\\
Scope:{\bf local}.\\
Type:{\bf optional}.\\
Intent: {\bf in}.\\
Default: true.
\end{description}

\begin{description}
\item[\bf On Return]
\item[info] Error code.\\
Scope: {\bf local} \\
Type: {\bf required} \\
Intent: {\bf out}.\\
An integer value; 0 means no error has been detected. 
\end{description}
{\par\noindent\large\bfseries Notes}
\begin{enumerate}
\item On exit from this routine the sparse matrix is in the update
  state. 
\end{enumerate}
%
%%   psb_spupdate %%
%
%% \subsection{psb\_spupdate --- Updates a sparse matrix.}

%% \syntax{call psb\_spupdate}{a, ia, ja, blck, desc\_a, info, ix, jx, updflag}

%% \begin{description}
%% \item[\bf On Entry]
%% \end{description}

%% \begin{description}
%% \item[\bf On Return]
%% \end{description}
%% %
%% %%   psb_csrp %%
%% %
%% \subsection{psb\_csrp --- Applies a right permutation to a sparse
%% matrix}

%% \syntax{call psb\_csrp}{trans, iperm, a, info}

%% \begin{description}
%% \item[\bf On Entry]
%% \item[trans] A character that specifies whether to permute $A$  or $A^T$.\\
%% Scope: {\bf local} \\
%% Type: {\bf required}\\
%% Specified as: a single character with value 'N' for $A$ or 'T' for $A^T$.\\
%% \item[iperm] An integer array containing permutation information.\\
%% Scope: {\bf local} \\
%% Type: {\bf required}\\
%% Specified as: an integer one-dimensional array.\\
%% \item[a] The sparse matrix to be permuted.\\
%% Scope: {\bf local} \\
%% Type: {\bf required}\\
%% Specified as: a \spdata variable.\\
%% \begin{description}
%% \item[\bf On Return]
%% \item[info] Error code.\\
%% Scope: {\bf local} \\
%% Type: {\bf required}\\
%% Specified as: Integer scalar.\\
%% \end{description}


%
%%   psb_alloc %%
%
\clearpage\subsection{psb\_geall --- Allocates a dense matrix}
%\addcontentsline{toc}{subsection}{psb\_geall}

\begin{verbatim}
call psb_geall(x, desc_a, info[, dupl, bldmode, n, lb])
\end{verbatim}

\begin{description}
\item[Type:] Synchronous.
\item[\bf On Entry]
\item[desc\_a] The communication descriptor.\\
Scope: {\bf local} \\
Type: {\bf required}\\
Intent: {\bf in}.\\
Specified as: a variable of type \descdata.\\
\item[n] The number of columns of the dense matrix to be allocated.\\
Scope: {\bf local} \\
Type: {\bf optional}\\
Intent: {\bf in}.\\
Specified as: Integer scalar, default $1$. It is not a valid argument  if $x$ is a
rank-1 array. 
\item[lb] The lower bound for the column index range of the dense matrix to be allocated.\\
Scope: {\bf local} \\
Type: {\bf optional}\\
Intent: {\bf in}.\\
Specified as: Integer scalar, default $1$. It is not a valid argument if $x$ is a
rank-1 array. 
\item[dupl] How to handle duplicate coefficients.\\
Scope: {\bf global}.\\
Type: {\bf optional}.\\
Intent: {\bf in}.\\
Specified as: integer, possible values: \verb|psb_dupl_ovwrt_|,
\verb|psb_dupl_add_|, \verb|psb_dupl_err_|.
\item[buildmode] Whether to keep track of matrix entries that do not
  belong to the current process.\\ 
Scope: {\bf global}.\\
Type: {\bf optional}.\\
Intent: {\bf in}.\\
Specified as: an integer value \verb|psb_matbld_noremote_|,
\verb|psb_matbld_remote_|.  Default:  \verb|psb_matbld_noremote_|.
\end{description}

\begin{description}
\item[\bf On Return]
\item[x] The dense matrix to be allocated.\\
Scope: {\bf local} \\
Type: {\bf required}\\
Intent: {\bf out}.\\
Specified as: a rank one or two array with the ALLOCATABLE  attribute
or an object of type \vdata, of type real, complex or integer.\\ 
\item[info] Error code.\\
Scope: {\bf local} \\
Type: {\bf required} \\
Intent: {\bf out}.\\
An integer value; 0 means no error has been detected. 
\end{description}


%
%%   psb_ins %%
%
\clearpage\subsection{psb\_geins --- Dense matrix insertion routine}
%\addcontentsline{toc}{subsection}{psb\_geins}

\begin{verbatim}
call psb_geins(m, irw, val, x, desc_a, info [,local])
\end{verbatim}

\begin{description}
\item[Type:] Asynchronous.
\item[\bf On Entry]
\item[m] Number of rows in $val$  to be inserted.\\
Scope:{\bf local}.\\
Type:{\bf required}.\\
Intent: {\bf in}.\\
Specified as: an integer value.
\item[irw] Indices of the rows to be inserted. Specifically, row $i$
  of $val$ will be inserted into the local row corresponding to the
  global row index $irw(i)$.
Scope:{\bf local}.\\
Type:{\bf required}.\\
Intent: {\bf in}.\\
Specified as: an integer array.
\item[val] the dense submatrix to be inserted.\\
Scope:{\bf local}.\\
Type:{\bf required}.\\
Intent: {\bf in}.\\
Specified as: a rank 1 or 2  array.
Specified as: an integer value.
\item[desc\_a] the communication descriptor.\\
Scope:{\bf local}.\\
Type:{\bf required}.\\
Intent: {\bf in}.\\
Specified as: a structured data of type \descdata.
\item[local] Whether the entries in the index vector \verb|irw|,
   are already in local  numbering. \\
 Scope:{\bf local}.\\
 Type:{\bf optional}.\\
 Specified as: a logical value; default: \verb|.false.|.

\end{description}

\begin{description}
\item[\bf On Return]
\item[x] the output dense matrix.\\
Scope: {\bf local} \\
Type: {\bf required}\\
Intent: {\bf inout}.\\
Specified as:  a rank one or two array or an object of type \vdata, of
type real, complex or integer.\\ 
\item[info] Error code.\\
Scope: {\bf local} \\
Type: {\bf required} \\
Intent: {\bf out}.\\
An integer value; 0 means no error has been detected. 
\end{description}

{\par\noindent\large\bfseries Notes}
\begin{enumerate}
\item Dense vectors/matrices do not have an associated state;
\item Duplicate entries are either overwritten or added, there is no
  provision for raising an error condition. 
\end{enumerate}


%
%%   psb_asb %%
%
\clearpage\subsection{psb\_geasb --- Assembly a dense matrix}
%\addcontentsline{toc}{subsection}{psb\_geasb}

\begin{verbatim}
call psb_geasb(x, desc_a, info, mold)
\end{verbatim}

\begin{description}
\item[Type:] Synchronous.
\item[\bf On Entry]
\item[desc\_a] The communication descriptor.\\
Scope: {\bf local} \\
Type: {\bf required}\\
Intent: {\bf in}.\\
Specified as: a variable of type \descdata.\\
\item[mold] The desired dynamic type for the internal vector storage.\\
Scope: {\bf local}.\\
Type: {\bf optional}.\\
Intent: {\bf in}.\\
Specified as: an object of a class derived from \vbasedata; this is
only allowed when $x$ is of type \vdata.
\end{description}

\begin{description}
\item[\bf On Return]
\item[x] The dense matrix to be assembled.\\
Scope: {\bf local} \\
Type: {\bf required}\\
Intent: {\bf inout}.\\
Specified as: a rank one or two array with the ALLOCATABLE or an
object of type \vdata, of type real, complex or integer.\\
\item[info] Error code.\\
Scope: {\bf local} \\
Type: {\bf required} \\
Intent: {\bf out}.\\
An integer value; 0 means no error has been detected. 
\end{description}
%
%%   psb_free %%
%
\clearpage\subsection{psb\_gefree --- Frees a dense matrix}
%\addcontentsline{toc}{subsection}{psb\_gefree}

\begin{verbatim}
call psb_gefree(x, desc_a, info)
\end{verbatim}

\begin{description}
\item[Type:] Synchronous.
\item[\bf On Entry]
\item[x] The dense matrix to
  be freed.\\
Scope: {\bf local} \\
Type: {\bf required}\\
Intent: {\bf inout}.\\
Specified as: a rank one or two array with the ALLOCATABLE or an
object of type \vdata, of type real, complex or integer.\\

\item[desc\_a] The communication descriptor.\\
Scope: {\bf local} \\
Type: {\bf required}\\
Intent: {\bf in}.\\
Specified as: a variable of type \descdata.\\
\end{description}

\begin{description}
\item[\bf On Return]
\item[info] Error code.\\
Scope: {\bf local} \\
Type: {\bf required} \\
Intent: {\bf out}.\\
An integer value; 0 means no error has been detected. 
\end{description}


%
%%   psb_gelp %%
%
\clearpage\subsection{psb\_gelp --- Applies a left permutation to a dense
  matrix}
%\addcontentsline{toc}{subsection}{psb\_gelp}

\begin{verbatim}
call psb_gelp(trans, iperm, x, info)
\end{verbatim}

\begin{description}
\item[Type:] Asynchronous.
\item[\bf On Entry]
\item[trans] A character that specifies whether to permute $A$  or $A^T$.\\
Scope: {\bf local} \\
Type: {\bf required}\\
Intent: {\bf in}.\\
Specified as: a single character with value 'N' for $A$ or 'T' for $A^T$.\\
\item[iperm] An integer array containing permutation information.\\
Scope: {\bf local} \\
Type: {\bf required}\\
Intent: {\bf in}.\\
Specified as: an integer one-dimensional array.\\
\item[x] The dense matrix to be permuted.\\
Scope: {\bf local} \\
Type: {\bf required}\\
Intent: {\bf inout}.\\
Specified as: a one or two dimensional array.\\
\end{description}

\begin{description}
\item[\bf On Return]
\item[info] Error code.\\
Scope: {\bf local} \\
Type: {\bf required} \\
Intent: {\bf out}.\\
An integer value; 0 means no error has been detected. 
\end{description}


%
%%   psb_glob_to_loc %%
%
\clearpage\subsection{psb\_glob\_to\_loc --- Global to local indices
  convertion}
%\addcontentsline{toc}{subsection}{psb\_glob\_to\_loc}

\begin{verbatim}
call psb_glob_to_loc(x, y, desc_a, info, iact,owned)
call psb_glob_to_loc(x, desc_a, info, iact,owned)
\end{verbatim}

\begin{description}
\item[Type:] Asynchronous.
\item[\bf On Entry]
\item[x] An integer vector of indices to be converted.\\
Scope: {\bf local} \\
Type: {\bf required}\\
Intent: {\bf in, inout}.\\
Specified as: a rank one integer array.\\
\item[desc\_a] the communication descriptor.\\
Scope:{\bf local}.\\
Type:{\bf required}.\\
Intent: {\bf in}.\\
Specified as: a structured data of type \descdata.
\item[iact] specifies action to be taken in case of range errors. 
Scope: {\bf global} \\
Type: {\bf optional}\\
Intent: {\bf in}.\\
Specified as: a character variable  \verb|I|gnore, \verb|W|arning or
\verb|A|bort, default \verb|I|gnore.
\item[owned] Specfies valid range of input 
Scope: {\bf global} \\
Type: {\bf optional}\\
Intent: {\bf in}.\\
If true, then only indices strictly owned by the current process are
considered valid, if false then halo indices are also
accepted. Default: false. 
\end{description}

\begin{description}
\item[\bf On Return]
\item[x] If $y$ is not present,
  then $x$ is overwritten with the translated integer indices. 
Scope: {\bf global} \\
Type: {\bf required}\\
Intent: {\bf inout}.\\
Specified as: a rank one integer array.
\item[y] If $y$ is  present,
  then $y$ is overwritten with the translated integer indices, and $x$
  is left unchanged. 
Scope: {\bf global} \\
Type: {\bf optional}\\
Intent: {\bf out}.\\
Specified as: a rank one integer array.
\item[info] Error code.\\
Scope: {\bf local} \\
Type: {\bf required} \\
Intent: {\bf out}.\\
An integer value; 0 means no error has been detected. 
\end{description}

{\par\noindent\large\bfseries Notes}
\begin{enumerate}
\item If an input index is out of range, then the corresponding output
  index is set to a negative number;
\item The default \verb|I|gnore means that the negative output is the
  only action taken on an out-of-range input.
\end{enumerate}


\clearpage\subsection{psb\_loc\_to\_glob --- Local to global indices
  conversion}
%\addcontentsline{toc}{subsection}{psb\_loc\_to\_glob}

\begin{verbatim}
call psb_loc_to_glob(x, y, desc_a, info, iact)
call psb_loc_to_glob(x, desc_a, info, iact)
\end{verbatim}

\begin{description}
\item[Type:] Asynchronous.
\item[\bf On Entry]
\item[x] An integer vector of indices to be converted.\\
Scope: {\bf local} \\
Type: {\bf required}\\
Intent: {\bf in, inout}.\\
Specified as: a rank one integer array.\\
\item[desc\_a] the communication descriptor.\\
Scope:{\bf local}.\\
Type:{\bf required}.\\
Intent: {\bf in}.\\
Specified as: a structured data of type \descdata.
\item[iact] specifies action to be taken in case of range errors. 
Scope: {\bf global} \\
Type: {\bf optional}\\
Intent: {\bf in}.\\
Specified as: a character variable  \verb|I|gnore, \verb|W|arning or
\verb|A|bort, default \verb|I|gnore.
\end{description}

\begin{description}
\item[\bf On Return]
\item[x] If $y$ is not present,
  then $x$ is overwritten with the translated integer indices. 
Scope: {\bf global} \\
Type: {\bf required}\\
Intent: {\bf inout}.\\
Specified as: a rank one integer array.
\item[y] If $y$ is not present,
  then $y$ is overwritten with the translated integer indices, and $x$
  is left unchanged. 
Scope: {\bf global} \\
Type: {\bf optional}\\
Intent: {\bf out}.\\
Specified as: a rank one integer array.
\item[info] Error code.\\
Scope: {\bf local} \\
Type: {\bf required} \\
Intent: {\bf out}.\\
An integer value; 0 means no error has been detected. 
\end{description}



%
%%   psb_loc_to_glob %%
%
\clearpage\subsection{psb\_is\_owned  --- }
%\addcontentsline{toc}{subsection}{psb\_is\_owned }

\begin{verbatim}
call psb_is_owned(x, desc_a)
\end{verbatim}

\begin{description}
\item[Type:] Asynchronous.
\item[\bf On Entry]
\item[x] Integer index.\\
Scope: {\bf local} \\
Type: {\bf required}\\
Intent: {\bf in}.\\
Specified as: a scalar integer.\\
\item[desc\_a] the communication descriptor.\\
Scope:{\bf local}.\\
Type:{\bf required}.\\
Intent: {\bf in}.\\
Specified as: a structured data of type \descdata.
\end{description}

\begin{description}
\item[\bf On Return]
\item[Function value] A logical mask which is true if 
  $x$ is  owned by the current process
Scope: {\bf local} \\
Type: {\bf required}\\
Intent: {\bf out}.\\
\end{description}


{\par\noindent\large\bfseries Notes}
\begin{enumerate}
\item This routine returns a \verb|.true.| value for an index
  that is strictly owned by the current process, excluding the halo
  indices
\end{enumerate}


\clearpage\subsection{psb\_owned\_index  --- }
%\addcontentsline{toc}{subsection}{psb\_owned\_index }

\begin{verbatim}
call psb_owned_index(y, x, desc_a, info)
\end{verbatim}

\begin{description}
\item[Type:] Asynchronous.
\item[\bf On Entry]
\item[x] Integer indices.\\
Scope: {\bf local} \\
Type: {\bf required}\\
Intent: {\bf in, inout}.\\
Specified as: a scalar or a rank one integer array.\\
\item[desc\_a] the communication descriptor.\\
Scope:{\bf local}.\\
Type:{\bf required}.\\
Intent: {\bf in}.\\
Specified as: a structured data of type \descdata.
\item[iact] specifies action to be taken in case of range errors. 
Scope: {\bf global} \\
Type: {\bf optional}\\
Intent: {\bf in}.\\
Specified as: a character variable  \verb|I|gnore, \verb|W|arning or
\verb|A|bort, default \verb|I|gnore.
\end{description}

\begin{description}
\item[\bf On Return]
\item[y] A logical mask which is true for all corresponding entries of
  $x$ that are owned by the current process
Scope: {\bf local} \\
Type: {\bf required}\\
Intent: {\bf out}.\\
Specified as: a scalar or rank one logical array.
\item[info] Error code.\\
Scope: {\bf local} \\
Type: {\bf required} \\
Intent: {\bf out}.\\
An integer value; 0 means no error has been detected. 
\end{description}


{\par\noindent\large\bfseries Notes}
\begin{enumerate}
\item This routine returns a \verb|.true.| value for those indices
  that are strictly owned by the current process, excluding the halo
  indices
\end{enumerate}


\clearpage\subsection{psb\_is\_local  --- }
%\addcontentsline{toc}{subsection}{psb\_is\_local }

\begin{verbatim}
call psb_is_local(x, desc_a)
\end{verbatim}

\begin{description}
\item[Type:] Asynchronous.
\item[\bf On Entry]
\item[x] Integer index.\\
Scope: {\bf local} \\
Type: {\bf required}\\
Intent: {\bf in}.\\
Specified as: a scalar integer.\\
\item[desc\_a] the communication descriptor.\\
Scope:{\bf local}.\\
Type:{\bf required}.\\
Intent: {\bf in}.\\
Specified as: a structured data of type \descdata.
\end{description}

\begin{description}
\item[\bf On Return]
\item[Function value] A logical mask which is true if 
  $x$ is  local to the current process
Scope: {\bf local} \\
Type: {\bf required}\\
Intent: {\bf out}.\\
\end{description}


{\par\noindent\large\bfseries Notes}
\begin{enumerate}
\item This routine returns a \verb|.true.| value for an index
  that is local to the current process, including the halo
  indices
\end{enumerate}

\clearpage\subsection{psb\_local\_index  --- }
%\addcontentsline{toc}{subsection}{psb\_local\_index }

\begin{verbatim}
call psb_local_index(y, x, desc_a, info)
\end{verbatim}

\begin{description}
\item[Type:] Asynchronous.
\item[\bf On Entry]
\item[x] Integer indices.\\
Scope: {\bf local} \\
Type: {\bf required}\\
Intent: {\bf in, inout}.\\
Specified as: a scalar or a rank one integer array.\\
\item[desc\_a] the communication descriptor.\\
Scope:{\bf local}.\\
Type:{\bf required}.\\
Intent: {\bf in}.\\
Specified as: a structured data of type \descdata.
\item[iact] specifies action to be taken in case of range errors. 
Scope: {\bf global} \\
Type: {\bf optional}\\
Intent: {\bf in}.\\
Specified as: a character variable  \verb|I|gnore, \verb|W|arning or
\verb|A|bort, default \verb|I|gnore.
\end{description}

\begin{description}
\item[\bf On Return]
\item[y] A logical mask which is true for all corresponding entries of
  $x$ that are local to  the current process
Scope: {\bf local} \\
Type: {\bf required}\\
Intent: {\bf out}.\\
Specified as: a scalar or rank one logical array.
\item[info] Error code.\\
Scope: {\bf local} \\
Type: {\bf required} \\
Intent: {\bf out}.\\
An integer value; 0 means no error has been detected. 
\end{description}


{\par\noindent\large\bfseries Notes}
\begin{enumerate}
\item This routine returns a \verb|.true.| value for those indices
  that are local to  the current process, including the halo
  indices.
\end{enumerate}



%
%%   psb_ins %%
%
\clearpage\subsection{psb\_get\_boundary --- Extract list of boundary
  elements}
%\addcontentsline{toc}{subsection}{psb\_get\_boundary}

\begin{verbatim}
call psb_get_boundary(bndel, desc, info)
\end{verbatim}

\begin{description}
\item[Type:] Asynchronous.
\item[\bf On Entry]
\item[desc] the communication descriptor.\\
Scope:{\bf local}.\\
Type:{\bf required}.\\
Intent: {\bf in}.\\
Specified as: a structured data of type \descdata.
\end{description}

\begin{description}
\item[\bf On Return]
\item[bndel] The list of boundary elements on the calling process, in
  local numbering.\\
Scope: {\bf local} \\
Type: {\bf required}\\
Intent: {\bf out}.\\
Specified as: a rank one array with the ALLOCATABLE
attribute, of type integer.\\
\item[info] Error code.\\
Scope: {\bf local} \\
Type: {\bf required} \\
Intent: {\bf out}.\\
An integer value; 0 means no error has been detected. 
\end{description}

{\par\noindent\large\bfseries Notes}
\begin{enumerate}
\item If there are no boundary elements (i.e., if the local part of
  the connectivity graph is self-contained) the output vector is set
  to the ``not allocated'' state. 
\item Otherwise the size of \verb|bndel| will be exactly equal to the
  number of boundary elements. 
\end{enumerate}

\clearpage\subsection{psb\_get\_overlap --- Extract list of overlap
  elements}
%\addcontentsline{toc}{subsection}{psb\_get\_overlap}

\begin{verbatim}
call psb_get_overlap(ovrel, desc, info)
\end{verbatim}

\begin{description}
\item[Type:] Asynchronous.
\item[\bf On Entry]
\item[desc] the communication descriptor.\\
Scope:{\bf local}.\\
Type:{\bf required}.\\
Intent: {\bf in}.\\
Specified as: a structured data of type \descdata.
\end{description}

\begin{description}
\item[\bf On Return]
\item[ovrel] The list of overlap elements on the calling process, in
  local numbering.\\
Scope: {\bf local} \\
Type: {\bf required}\\
Intent: {\bf out}.\\
Specified as: a rank one array with the ALLOCATABLE
attribute, of type integer.\\
\item[info] Error code.\\
Scope: {\bf local} \\
Type: {\bf required} \\
Intent: {\bf out}.\\
An integer value; 0 means no error has been detected. 
\end{description}

{\par\noindent\large\bfseries Notes}
\begin{enumerate}
\item If there are no overlap elements the output vector is set
  to the ``not allocated'' state.  
\item Otherwise the size of \verb|ovrel| will be exactly equal to the
  number of overlap elements. 
\end{enumerate}



\clearpage\subsection{psb\_sp\_getrow --- Extract row(s) from a sparse
  matrix}
%\addcontentsline{toc}{subsection}{psb\_sp\_getrow}

\begin{verbatim}
call psb_sp_getrow(row, a, nz, ia, ja, val, info, &
              & append, nzin, lrw)
\end{verbatim}

\begin{description}
\item[Type:] Asynchronous.
\item[\bf On Entry]
\item[row] The (first) row to be extracted.\\
Scope:{\bf local}\\
Type:{\bf required}\\
Intent: {\bf in}.\\
Specified as: an integer $>0$.
\item[a] the matrix from  which to get rows.\\
Scope:{\bf local}\\
Type:{\bf required}\\
Intent: {\bf in}.\\
Specified as: a structured data of type \spdata.
\item[append] Whether to append or overwrite existing output.\\
Scope:{\bf local}\\
Type:{\bf optional}\\
Intent: {\bf in}.\\
Specified as: a logical value default: false (overwrite).
\item[nzin] Input size to be appended to.\\
Scope:{\bf local}\\
Type:{\bf optional}\\
Intent: {\bf in}.\\
Specified as: an integer $>0$. When append is true, specifies how many
entries in the output vectors are already filled. 
\item[lrw] The last  row to be extracted.\\
Scope:{\bf local}\\
Type:{\bf optional}\\
Intent: {\bf in}.\\
Specified as: an integer $>0$, default: $row$.

%% \item[is] the starting row on matrix $a$.\\
%% Scope:{\bf local}.\\
%% Type:{\bf optional}.\\
%% Specified as: an integer vaule.
%% \item[js] the starting column on matrix $a$.\\
%% Scope:{\bf local}.\\
%% Type:{\bf optional}\\
%% Specified as: an integer value
\end{description}

\begin{description}
\item[\bf On Return]
\item[nz] the number of elements returned by this call.\\
Scope:{\bf local}.\\
Type:{\bf required}.\\
Intent: {\bf out}.\\
Returned  as: an integer scalar.
\item[ia] the row indices.\\
Scope:{\bf local}.\\
Type:{\bf required}.\\
Intent: {\bf inout}.\\
Specified as: an integer array with the \verb|ALLOCATABLE| attribute.
\item[ja] the column indices of the elements to be inserted.\\
Scope:{\bf local}.\\
Type:{\bf required}.\\
Intent: {\bf inout}.\\
Specified as: an integer array with the \verb|ALLOCATABLE| attribute.
\item[val] the elements to be inserted.\\
Scope:{\bf local}.\\
Type:{\bf required}.\\
Intent: {\bf inout}.\\
Specified as: a real  array with the \verb|ALLOCATABLE| attribute.
\item[info] Error code.\\
Scope: {\bf local} \\
Type: {\bf required} \\
Intent: {\bf out}.\\
An integer value; 0 means no error has been detected. 
\end{description}

{\par\noindent\large\bfseries Notes}
\begin{enumerate}
\item The output $nz$ is always the size of the output generated by
  the current call; thus, if \verb|append=.true.|, the total output
  size will be $nzin+nz$, with the newly extracted coefficients stored in
  entries \verb|nzin+1:nzin+nz| of the array arguments;
\item When \verb|append=.true.|  the output arrays are reallocated as
  necessary;
\item The row and column indices are returned in the local numbering
  scheme; if the global numbering is desired, the user may employ the
  \verb|psb_loc_to_glob| routine on the output.
\end{enumerate}




\clearpage\subsection{psb\_sizeof --- Memory occupation}
%\addcontentsline{toc}{subsection}{psb\_sizeof}

This function computes the memory occupation of a PSBLAS object.


\begin{verbatim}
isz = psb_sizeof(a)
isz = psb_sizeof(desc_a)
isz = psb_sizeof(prec)
\end{verbatim}

\begin{description}
\item[Type:] Asynchronous.
\item[\bf On Entry]
\item[a] A sparse matrix
$A$. \\ 
Scope: {\bf local} \\
Type: {\bf required}\\
Intent: {\bf in}.\\
Specified as: a structured data of type \spdata.
\item[desc\_a] Communication descriptor.\\
Scope: {\bf local} \\
Type: {\bf required}\\
Intent: {\bf in}.\\
Specified as: a structured data of type \descdata.
\item[prec] 
Scope: {\bf local} \\
Type: {\bf required}\\
Intent: {\bf in}.\\
Specified as: a preconditioner data structure \precdata.
\item[\bf On Return] 
\item[Function value] The memory occupation of the object specified in
  the calling sequence, in bytes.\\
Scope: {\bf local} \\
Returned  as: an \verb|integer(psb_long_int_k_)| number.
\end{description}


\clearpage\subsection{Sorting utilities --- }
%\addcontentsline{toc}{subsection}{Sorting utilities}

{\par\noindent\large\bfseries psb\_msort --- Sorting by the Merge-sort
  algorithm}

{\par\noindent\large\bfseries psb\_qsort --- Sorting by the Quicksort
  algorithm}

{\par\noindent\large\bfseries psb\_hsort --- Sorting by the Heapsort  algorithm}
\begin{verbatim}
call psb_msort(x,ix,dir,flag)
call psb_qsort(x,ix,dir,flag)
call psb_hsort(x,ix,dir,flag)
\end{verbatim}

These  serial routines sort a sequence $X$ into ascending or
descending  order. The argument meaning is identical for the three
calls; the only difference is the algorithm used to accomplish the
task (see Usage Notes below). 
\begin{description}
\item[Type:] Asynchronous.
\item[\bf  On Entry ]
\item[x] The sequence to be sorted.\\
Type:{\bf required}.\\
Specified as: an integer, real or complex  array of rank 1.
\item[ix] A vector of indices.\\
Type:{\bf optional}.\\
Specified as: an integer array of (at least) the same size as $X$.
\item[dir] The desired ordering.\\
Type:{\bf optional}.\\
Specified as: an integer value: \begin{description}
\item[Integer and real data:] \verb|psb_sort_up_|,
\verb|psb_sort_down_|, \verb|psb_asort_up_|, \verb|psb_asort_down_|;
default \verb|psb_sort_up_|. 
\item[Complex data:] \verb|psb_lsort_up_|,
\verb|psb_lsort_down_|, \verb|psb_asort_up_|, \verb|psb_asort_down_|;
default \verb|psb_lsort_up_|. 
\end{description}
\item[flag] Whether to keep the original values in $IX$.\\
Type:{\bf optional}.\\
Specified as: an integer value \verb|psb_sort_ovw_idx_| or
\verb|psb_sort_keep_idx_|; default \verb|psb_sort_ovw_idx_|.

\end{description}

\begin{description}
\item[\bf On Return]
\item[x] The sequence of values, in the chosen ordering.\\ 
Type:{\bf required}.\\
Specified as: an integer, real or complex array of rank 1.
\item[ix] A vector of indices.\\
Type: {\bf Optional} \\
An integer array of rank 1, whose entries are moved to the same
position as the corresponding entries in $x$.
\end{description}

{\goodbreak\par\noindent\large\bfseries Notes}
\begin{enumerate}
\item For integer or real data the sorting can be performed in the up/down direction, on the
  natural or absolute values;
\item For  complex data the sorting can  be done in a lexicographic
  order (i.e.: sort on the real part with ties broken according to
  the imaginary part) or on the absolute values; 
\item The routines return the items in the chosen ordering; the
  output difference is the handling of  ties (i.e. items with an
  equal   value) in the original input. With the merge-sort algorithm
  ties are   preserved in the same relative order as they had in the
  original   sequence,  while this is not guaranteed for quicksort or
  heapsort; 
\item If $flag = psb\_sort\_ovw\_idx\_$ then the entries in $ix(1:n)$
  where $n$ is the size of $x$ are initialized to $ix(i) \leftarrow
  i$; thus, upon return from the subroutine, for each
  index $i$ we have in $ix(i)$ the position that the item $x(i)$
  occupied in the original data sequence;
\item If $flag = psb\_sort\_keep\_idx\_$  the routine will assume that
  the entries in $ix(:)$ have already been initialized by the user;
\item The three sorting algorithms have a similar $O(n \log n)$ expected
  running time; in the average case quicksort will be the
  fastest and merge-sort the slowest. However note that:
\begin{enumerate}
\item The worst case running time for quicksort is $O(n^2)$; the algorithm
  implemented here follows the well-known median-of-three heuristics,
  but the worst case may still apply;
\item The worst case running time for merge-sort and heap-sort is
  $O(n\log n)$  as the average case;
\item The merge-sort algorithm is implemented to take advantage of 
  subsequences that may be already in the desired ordering prior to
  the subroutine call; this situation is relatively common when
  dealing with groups of indices of sparse matrix entries, thus
  merge-sort is the preferred  choice when a sorting is needed
  by other routines in the library. 
\end{enumerate}
\end{enumerate}



%%% Local Variables: 
%%% mode: latex
%%% TeX-master: "userguide"
%%% End: 
