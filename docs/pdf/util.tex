\section{Utilities}
\label{sec:util}


\subroutine{}{Sorting utilities}

\subroutine*{psb\_msort}{Sorting by the Merge-sort algorithm}
\subroutine*{psb\_qsort}{Sorting by the Quicksort  algorithm}
\syntax{call psb\_msort}{x,ix,dir,flag}
\syntax*{call psb\_qsort}{x,ix,dir,flag}

These  serial routines sort a sequence $X$ into ascending or
descending  order. The argument meaning is identical for the two
calls; the only difference is the algorithm used to accomplish the
task (see Usage Notes below). 
\begin{description}
\item[\bf  On Entry ]
\item[x] The sequence to be sorted.\\
Type:{\bf required}.\\
Specified as: an integer array of rank 1.
\item[ix] A vector of indices.\\
Type:{\bf optional}.\\
Specified as: an integer array of (at least) the same size as $X$.
\item[dir] The desired ordering.\\
Type:{\bf optional}.\\
Specified as: an integer value \verb|psb_sort_up_| or
\verb|psb_sort_down_|; default \verb|psb_sort_up_|.
\item[flag] Whether to keep the original values in $IX$.\\
Type:{\bf optional}.\\
Specified as: an integer value \verb|psb_sort_ovw_idx_| or
\verb|psb_sort_keep_idx_|; default \verb|psb_sort_ovw_idx_|.

\end{description}

\begin{description}
\item[\bf On Return]
\item[x] The sequence of values, in the chosen ordering.\\ 
Type:{\bf required}.\\
Specified as: an integer array of rank 1.
\item[ix] A vector of indices.\\
Type: {\bf Optional} \\
An integer array of rank 1, whose entries are moved to the same
position as the corresponding entries in $x$.
\end{description}
\section*{Usage notes}
\begin{enumerate}
\item The two routines return the items in the chosen ordering; the
  only output difference is the handling of  ties (i.e. items with an
  equal   value) in the original input. With the merge-sort algorithm
  ties are   preserved in the same order as they had in the original
  sequence,   while this is not guaranteed for quicksort 
\item If $flag = psb\_sort\_ovw\_idx\_$ then the entries in $ix(1:n)$
  where $n$ is the size of $x$ are initialized to $ix(i) \leftarrow
  i$; thus, upon return from the subroutine, for each
  index $i$ we have in $ix(i)$ the position that the item $x(i)$
  occupied in the original data sequence;
\item If $flag = psb\_sort\_keep\_idx\_$  the routine will assume that
  the entries in $ix(:)$ have already been initialized by the user;
\item The two sorting algorithms have a similar $O(n \log n)$ expected
  running time; in the average case quicksort will be the
  fastest. However note that:
\begin{enumerate}
\item The worst case running time for quicksort is $O(n^2)$; the algorithm
  implemented here follows the well-known median-of-three heuristics,
  but the worst case may still apply;
\item The worst case running time for merge-sort is the same as the
  average case;
\item The merge-sort algorithm is implemented to take advantage of 
  subsequences that may be  already in the desired ordering at the
  beginning; this situation is relatively common when dealing with
  indices of sparse matrix entries, thus merge-sort  is the
  preferred   choice when a sorting is needed by other routines in the
  library. 
\end{enumerate}
\end{enumerate}


%\subroutine{PSB\_HBIO\_MOD}{Input/Output in Harwell-Boeing format}

\subroutine*{hb\_read}{Read a sparse matrix from a file}
\syntax{call hb\_read}{a, iret, iunit, filename, b, mtitle}
 
\begin{description}
\item[\bf  On Entry ]
\item[filename] The name of the file to be read.\\
Type:{\bf optional}.\\
Specified as: a character variable containing a valid file name, or
\verb|-|, in which case the default input unit  5 (i.e. standard input
in Unix jargon) is used. Default: \verb|-|. 
\item[iunit] The Fortran file unit number.\\
Type:{\bf optional}.\\
Specified as: an integer value. Only meaningful if filename is not \verb|-|.
\end{description}

\begin{description}
\item[\bf On Return]
\item[a] the sparse matrix read from file.\\
Type:{\bf required}.\\
Specified as: a structured data of type \spdata.
\item[b] Rigth hand side.\\
Type: {\bf Optional} \\
An  array of type real or complex, rank 1 and having the ALLOCATABLE
attribute; will be allocated and filled in if the input file contains
a right hand side. 
\item[mtitle] Matrix title.\\
Type: {\bf Optional} \\
A charachter variable of length 72 holding a copy of the
matrix title as specified by the Harwell-Boeing format and contained
in the input file.
\item[iret] Error code.\\
Type: {\bf required} \\
An integer value; 0 means no error has been detected. 
\end{description}



\subroutine*{hb\_write}{Write a sparse matrix to a  file}
\syntax{call hb\_write}{a, iret, iunit, filename, key, rhs, mtitle}



\begin{description}
\item[\bf  On Entry ]
\item[a] the sparse matrix to be written.\\
Type:{\bf required}.\\
Specified as: a structured data of type \spdata.
\item[b] Rigth hand side.\\
Type: {\bf Optional} \\
An  array of type real or complex, rank 1 and having the ALLOCATABLE
attribute; will be allocated and filled in if the input file contains
a right hand side. 
\item[filename] The name of the file to be written to.\\
Type:{\bf optional}.\\
Specified as: a character variable containing a valid file name, or
\verb|-|, in which case the default output unit  6 (i.e. standard output
in Unix jargon) is used. Default: \verb|-|. 
\item[iunit] The Fortran file unit number.\\
Type:{\bf optional}.\\
Specified as: an integer value. Only meaningful if filename is not \verb|-|.
\item[key] Matrix key.\\
Type: {\bf Optional} \\
A charachter variable of length 8 holding the
matrix key as specified by the Harwell-Boeing format and to be
written to file.
\item[mtitle] Matrix title.\\
Type: {\bf Optional} \\
A charachter variable of length 72 holding the
matrix title as specified by the Harwell-Boeing format and to be
written to file.
\end{description}

\begin{description}
\item[\bf On Return]
\item[iret] Error code.\\
Type: {\bf required} \\
An integer value; 0 means no error has been detected. 
\end{description}




%\subroutine{PSB\_MMIO\_MOD}{Input/Output in MatrixMarket format}

\subroutine*{mm\_mat\_read}{Read a sparse matrix from a file}
\syntax{call mm\_mat\_read}{a, iret, iunit, filename}

\begin{description}
\item[\bf  On Entry ]
\item[filename] The name of the file to be read.\\
Type:{\bf optional}.\\
Specified as: a character variable containing a valid file name, or
\verb|-|, in which case the default input unit  5 (i.e. standard input
in Unix jargon) is used. Default: \verb|-|. 
\item[iunit] The Fortran file unit number.\\
Type:{\bf optional}.\\
Specified as: an integer value. Only meaningful if filename is not \verb|-|.
\end{description}

\begin{description}
\item[\bf On Return]
\item[a] the sparse matrix read from file.\\
Type:{\bf required}.\\
Specified as: a structured data of type \spdata.
\item[iret] Error code.\\
Type: {\bf required} \\
An integer value; 0 means no error has been detected. 
\end{description}



\subroutine*{mm\_mat\_write}{Write a sparse matrix to a  file}
\syntax{call mm\_mat\_write}{a, mtitle, iret, iunit, filename}
\begin{description}
\item[\bf  On Entry ]
\item[a] the sparse matrix to be written.\\
Type:{\bf required}.\\
Specified as: a structured data of type \spdata.
\item[mtitle] Matrix title.\\
Type: {\bf required} \\
A charachter variable holding a descriptive title for the matrix to be
 written to file.
\item[filename] The name of the file to be written to.\\
Type:{\bf optional}.\\
Specified as: a character variable containing a valid file name, or
\verb|-|, in which case the default output unit  6 (i.e. standard output
in Unix jargon) is used. Default: \verb|-|. 
\item[iunit] The Fortran file unit number.\\
Type:{\bf optional}.\\
Specified as: an integer value. Only meaningful if filename is not \verb|-|.
\end{description}

\begin{description}
\item[\bf On Return]
\item[iret] Error code.\\
Type: {\bf required} \\
An integer value; 0 means no error has been detected. 
\end{description}



%%% Local Variables: 
%%% mode: latex
%%% TeX-master: "userguide"
%%% End: 
