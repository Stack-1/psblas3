\section{Data management routines}
\label{sec:toolsrout}


%
%%   psb_cdall %%
%
\subroutine{psb\_cdall}{Allocates a communication descriptor}

\syntax{call psb\_cdall}{icontxt, desc\_a, info,mg=mg,parts=parts}
\syntax*{call psb\_cdall}{icontxt, desc\_a, info,vg=vg,flag=flag}
\syntax*{call psb\_cdall}{icontxt, desc\_a, info,vl=vl}
\syntax*{call psb\_cdall}{icontxt, desc\_a, info,nl=nl}

This subroutine initializes the communication descriptor associated
with an index space. Exactly one of the  optional arguments
\verb|parts|, \verb|vg|, \verb|vl| or \verb|nl| 
must be specified, thereby choosing
the specific initialization strategy:
\begin{description}
\item[\bf  On Entry ]
\item[Type:] Synchronous.
\item[icontxt] the communication context.\\
Scope:{\bf global}.\\
Type:{\bf required}.\\
Intent: {\bf in}.\\
Specified as: an integer value.
\item[vg] Data allocation: each index $i\in \{1\dots mg\}$ is allocated
  to process $vg(i)$.
Scope:{\bf global}.\\
Type:{\bf optional}.\\
Intent: {\bf in}.\\
Specified as: an integer array. 
\item[flag] Specifies whether entries in $vg$ are zero- or one-based.
Scope:{\bf global}.\\
Type:{\bf optional}.\\
Intent: {\bf in}.\\
Specified as: an integer value $0,1$, default $0$.

\item[mg] the (global) number of rows of the problem.\\
Scope:{\bf global}.\\
Type:{\bf optional}.\\
Intent: {\bf in}.\\
Specified as: an integer value. It is required if \verb|parts| is
specified. 
\item[parts] the subroutine that defines the partitioning scheme.\\
Scope:{\bf global}.\\
Type:{\bf required}.\\
Specified as: a subroutine. 
\item[vl] Data allocation: the set of global indices belonging to the
  calling process. 
Scope:{\bf local}.\\
Type:{\bf optional}.\\
Intent: {\bf in}.\\
Specified as: an integer array. 
\item[nl] Data allocation: in a generalized block-row distribution the
  number of indices belonging to the current process. 
Scope:{\bf local}.\\
Type:{\bf optional}.\\
Intent: {\bf in}.\\
Specified as: an integer value. 
\end{description}

\begin{description}
\item[\bf On Return]
\item[desc\_a] the communication descriptor.\\
Scope:{\bf local}.\\
Type:{\bf required}.\\
Intent: {\bf out}.\\
Specified as: a structured data of type \descdata.
\item[info] Error code.\\
Scope: {\bf local} \\
Type: {\bf required} \\
Intent: {\bf out}.\\
An integer value; 0 means no error has been detected. 
\end{description}

\section*{Notes}
\begin{enumerate}
\item Exactly one of the optional arguments  \verb|parts|, \verb|vg|,
  \verb|vl|, \verb|nl|  must be specified, thereby choosing the
  initialization   strategy as follows:
\begin{description}
\item[parts] In this case we have a subroutine specifying the mapping
  between global indices and process/local index pairs. If this
  optional argument   is specified, then it is mandatory to 
  specify the argument \verb|mg| as well.  
  The subroutine must conform to the following interface: 
\begin{verbatim}
  interface 
     subroutine psb_parts(glob_index,mg,np,pv,nv)
       integer, intent (in)  :: glob_index,np,mg
       integer, intent (out) :: nv, pv(*)
     end subroutine psb_parts
  end interface
\end{verbatim}
  The input arguments are:
  \begin{description}
    \item[glob\_index] The global index to be mapped;
    \item[np] The number of processes in the mapping;
    \item[mg] The total number of global rows in the mapping;
  \end{description}
  The output arguments are:
  \begin{description}
    \item[nv] The number of entries in \verb|pv|;
    \item[pv] A vector containint the indices of the processes to
    which the global index should be assigend; each entry must satisfy
    $0\le pv(i) < np$; if $nv>1$ we have an index assigned to multiple
    processes, i.e. we have an overlap among the subdomains.
  \end{description}
\item[vg] In this case the association between an index and a process
  is specified via an integer vector; the size of the index space is
  equal to the size of \verb|vg|, and each index $i$ is assigned to
  the process $vg(i)$. The vector \verb|vg| must be identical on all
  calling processes; its entries may have the ranges $(0\dots np-1)$
  or $(1\dots np)$ according to the value of \verb|flag|.
\item[vl] In this case we are specifying the list of indices assigned
  to the current process; thus, the global problem size $mg$ is given by
  the sum of the sizes of the individual vectors \verb|vl| specified
  on the calling processes. The subroutine will check that each entry
  in the global index space $(1\dots mg)$ is specified exactly once. 
\item[nl] In this case we are implying a generalized row-block
  distribution in which each process $I$ gets assigned a consecutive
  chunk of $N_I=nl$ global indices.
\end{description}
\item On exit from this routine the descriptor is in the build state
\end{enumerate}


%
%%   psb_cdins %%
%
\subroutine{psb\_cdins}{Communication descriptor insert routine}

\syntax{call psb\_cdins}{nz, ia, ja, desc\_a, info}

This subroutine examines the edges of the graph associated with the
discretization mesh (and isomorphic to the sparsity pattern of a
linear system coefficient matrix), storing them as necessary into the
communication descriptor.

\begin{description}
\item[Type:] Asynchronous.
\item[\bf On Entry]
\item[nz] the number of points being inserted.\\
Scope: {\bf local}.\\
Type: {\bf required}.\\
Intent: {\bf in}.\\
Specified as: an integer value.
\item[ia] the indices of the starting vertex of the edges  being inserted.\\
Scope: {\bf local}.\\
Type: {\bf required}.\\
Intent: {\bf in}.\\
Specified as: an integer array of length $nz$.
\item[ja]  the indices of the end vertex of the edges  being inserted.\\
Scope: {\bf local}.\\
Type: {\bf required}.\\
Intent: {\bf in}.\\
Specified as: an integer array of length $nz$.
%% \item[is] the row offset.\\
%% Scope:{\bf local}.\\
%% Type:{\bf optional}.\\
%% Specified as: an integer value.
%% \item[js] the column offset.\\
%% Scope: {\bf local}.\\
%% Type: {\bf optional}.\\
%% Specified as: an integer value.
\end{description}

\begin{description}
\item[\bf On Return]
\item[desc\_a] the updated communication descriptor.\\
Scope:{\bf local}.\\
Type:{\bf required}.\\
Intent: {\bf inout}.\\
Specified as: a structured data of type \descdata.
\item[info] Error code.\\
Scope: {\bf local} \\
Type: {\bf required} \\
Intent: {\bf out}.\\
An integer value; 0 means no error has been detected. 
\end{description}
\section*{Notes}
\begin{enumerate}
\item This routine may only be called if the descriptor is in the
  build state;
\item  This routine automatically ignores edges that do not
insist on the  current process, i.e. edges for which neither the starting
nor the end vertex belong to the current process. 
\end{enumerate}



%
%%   psb_cdasb %%
%
\subroutine{psb\_cdasb}{Communication descriptor assembly routine}

\syntax{call psb\_cdasb}{desc\_a, info}

\begin{description}
\item[Type:] Synchronous.
\item[\bf On Entry]
\item[desc\_a] the communication descriptor.\\
Scope:{\bf local}.\\
Type:{\bf required}.\\
Intent: {\bf inout}.\\
Specified as: a structured data of type \descdata.
\end{description}

\begin{description}
\item[\bf On Return]
\item[desc\_a] the communication descriptor.\\
Scope:{\bf local}.\\
Type:{\bf required}.\\
Intent: {\bf inout}.\\
Specified as: a structured data of type \descdata.
\item[info] Error code.\\
Scope: {\bf local} \\
Type: {\bf required} \\
Intent: {\bf out}.\\
An integer value; 0 means no error has been detected. 
%\item[arg] 
\end{description}
\section*{Notes}
\begin{enumerate}
\item On exit from this routine the descriptor is in the assembled
  state. 
\end{enumerate}



%
%%   psb_cdcpy %%
%
\subroutine{psb\_cdcpy}{Copies a communication descriptor}

\syntax{call psb\_cdcpy}{desc\_in, desc\_out, info}

\begin{description}
\item[Type:] Asynchronous.
\item[\bf On Entry]
\item[desc\_in] the communication descriptor.\\
Scope:{\bf local}.\\
Type:{\bf required}.\\
Intent: {\bf in}.\\
Specified as: a structured data of type \descdata.

\end{description}

\begin{description}
\item[\bf On Return]
\item[desc\_out] the communication descriptor copy.\\
Scope:{\bf local}.\\
Type:{\bf required}.\\
Intent: {\bf out}.\\
Specified as: a structured data of type \descdata.
\item[info] Error code.\\
Scope: {\bf local} \\
Type: {\bf required} \\
Intent: {\bf out}.\\
An integer value; 0 means no error has been detected. 
\end{description}


%
%%   psb_cdfree %%
%
\subroutine{psb\_cdfree}{Frees a communication descriptor}

\syntax{call psb\_cdfree}{desc\_a, info}

\begin{description}
\item[Type:] Synchronous.
\item[\bf On Entry]
\item[desc\_a] the communication descriptor to be freed.\\
Scope:{\bf local}.\\
Type:{\bf required}.\\
Intent: {\bf inout}.\\
Specified as: a structured data of type \descdata.
\end{description}

\begin{description}
\item[\bf On Return]
\item[info] Error code.\\
Scope: {\bf local} \\
Type: {\bf required} \\
Intent: {\bf out}.\\
An integer value; 0 means no error has been detected. 
\end{description}



%
%%   psb_cdcpy %%
%
\subroutine{psb\_cdbldext}{Build an extended  communication descriptor}

\syntax{call psb\_cdbldext}{a,desc\_a,nl,desc\_out, info, extype}

This subroutine builds an extended communication descriptor, based on
the input descriptor \verb|desc_a| and on the stencil specified
through the input sparse matrix \verb|a|. 
\begin{description}
\item[Type:] Synchronous.
\item[\bf On Entry]
\item[a] A sparse matrix
Scope:{\bf local}.\\
Type:{\bf required}.\\
Intent: {\bf in}.\\
Specified as: a structured data type.
\item[desc\_a] the communication descriptor.\\
Scope:{\bf local}.\\
Type:{\bf required}.\\
Intent: {\bf in}.\\
Specified as: a structured data of type \spdata.
\item[nl] the number of additional layers desired.\\
Scope:{\bf global}.\\
Type:{\bf required}.\\
Intent: {\bf in}.\\
Specified as: an integer value $nl\ge 0$. 
\item[extype] the kind of estension required.\\
Scope:{\bf global}.\\
Type:{\bf optional }.\\
Intent: {\bf in}.\\
Specified as: an integer value
\verb|psb_ovt_xhal_|, \verb|psb_ovt_asov_|, default: \verb|psb_ovt_xhal_|

\end{description}

\begin{description}
\item[\bf On Return]
\item[desc\_out] the extended communication descriptor.\\
Scope:{\bf local}.\\
Type:{\bf required}.\\
Intent: {\bf inout}.\\
Specified as: a structured data of type \descdata.
\item[info] Error code.\\
Scope: {\bf local} \\
Type: {\bf required} \\
Intent: {\bf out}.\\
An integer value; 0 means no error has been detected. 
\end{description}

\section*{Notes}
\begin{enumerate}
\item Specifying \verb|psb_ovt_xhal_| for the \verb|extype| argument
  the user will obtain a descriptor for a domain partition in which
  the additional layers are fetched as part of an (extended) halo;
  however the index-to-process mapping is identical to that of the
  base descriptor;
\item Specifying \verb|psb_ovt_asov_| for the \verb|extype| argument
  the user will obtain a descriptor with an overlapped decomposition:
  the additional layer is aggregated to  the local subdomain (and thus
  is an overlap), and a new halo extending beyond the last additional
  layer is formed. 
\end{enumerate}


%% %
%% %%   psb_cdren %%
%% %
%% \subroutine{psb\_cdren}{Applies a renumeration to a communication descriptor}

%% \syntax{call psb\_cdren}{trans, iperm, desc\_a, info}

%% \begin{description}
%% \item[\bf On Entry]
%% \item[Type:] Asynchronous.
%% \item[trans] A character that specifies whether to permute $A$  or $A^T$.\\
%% Scope: {\bf local} \\
%% Type: {\bf required}\\
%% Specified as: a single character with value 'N' for $A$ or 'T' for $A^T$.\\
%% \item[iperm] An integer array containing permutation information.\\
%% Scope: {\bf local} \\
%% Type: {\bf required}\\
%% Specified as: an integer one-dimensional array.\\
%% \item[desc\_a] the communication descriptor.\\
%% Scope:{\bf local}.\\
%% Type:{\bf required}.\\
%% Specified as: a structured data of type \descdata.
%% \end{description}

%% \begin{description}
%% \item[\bf On Return]
%% \item[info] Error code.
%% Scope: {\bf local} \\
%% Type: {\bf required}\\
%% Specified as: an integer variable.
%% \end{description}




%
%%   psb_descprt %%
%
%% \subroutine{psb\_cdprt}{Prints a descriptor}

%% \syntax{call psb\_cdprt}{iout, desc\_a, glob, short}

%% \begin{description}
%% \item[Type:] Asynchronous.
%% \item[\bf On Entry]
%% \item[iout] An integer that defines the output unit.
%% Scope: {\bf local} \\
%% Type: {\bf required}\\
%% Specified as: Integer scalar.\\
%% \item[desc\_a] The communication descriptor of type \descdata that
%%   must be printed.\\
%% Scope: {\bf local} \\
%% Type: {\bf required}\\
%% Specified as: a variable of type \descdata.\\
%% \end{description}

%% \begin{description}
%% \item[\bf On Return]
%% \item[glob] ??????
%% \item[short] ??????
%% \end{description}



%
%%   psb_spalloc %%
%
\subroutine{psb\_spall}{Allocates a sparse matrix}

\syntax{call psb\_spall}{a, desc\_a, info, nnz}

\begin{description}
\item[Type:] Synchronous.
\item[\bf On Entry]
\item[desc\_a] the communication descriptor.\\
Scope:{\bf local}.\\
Type:{\bf required}.\\
Intent: {\bf in}.\\
Specified as: a structured data of type \descdata.
\item[nnz] An estimate of the number of nonzeroes in the local
  part of the assembled matrix.\\ 
Scope: {\bf global}.\\
Type: {\bf optional}.\\
Intent: {\bf in}.\\
Specified as: an integer value. 
\end{description}

\begin{description}
\item[\bf On Return]
\item[a] the matrix to be allocated.\\
Scope:{\bf local}\\
Type:{\bf required}\\
Intent: {\bf out}.\\
Specified as: a structured data of type \spdata.
\item[info] Error code.\\
Scope: {\bf local} \\
Type: {\bf required} \\
Intent: {\bf out}.\\
An integer value; 0 means no error has been detected. 
\end{description}
\section*{Notes}
\begin{enumerate}
\item On exit from this routine the sparse matrix  is in the build
  state.
\item The descriptor may be in either the build or assembled state.
\item Providing a good estimate for the number of nonzeroes $nnz$ in
  the assembled matrix may substantially improve performance in the
  matrix build phase, as it will reduce or eliminate the need for
  (potentially multiple) data reallocations. 
\end{enumerate}



%
%%   psb_spins %%
%
\subroutine{psb\_spins}{Insert a cloud of elements into a sparse matrix}

\syntax{call psb\_spins}{nz, ia, ja, val, a, desc\_a, info}

\begin{description}
\item[Type:] Asynchronous.
\item[\bf On Entry]
\item[nz] the number of elements to be inserted.\\
Scope:{\bf local}.\\
Type:{\bf required}.\\
Intent: {\bf in}.\\
Specified as: an integer scalar.
\item[ia] the row indices of the elements to be inserted.\\
Scope:{\bf local}.\\
Type:{\bf required}.\\
Intent: {\bf in}.\\
Specified as: an integer array of size $nz$.
\item[ja] the column indices of the elements to be inserted.\\
Scope:{\bf local}.\\
Type:{\bf required}.\\
Intent: {\bf in}.\\
Specified as: an integer array of size $nz$.
\item[val] the elements to be inserted.\\
Scope:{\bf local}.\\
Type:{\bf required}.\\
Intent: {\bf in}.\\
Specified as: an array of size $nz$.
\item[desc\_a] The communication descriptor.\\
Scope: {\bf local}. \\
Type: {\bf required}.\\
Intent: {\bf inout}.\\
Specified as: a variable of type \descdata.\\
%% \item[is] the starting row on matrix $a$.\\
%% Scope:{\bf local}.\\
%% Type:{\bf optional}.\\
%% Specified as: an integer vaule.
%% \item[js] the starting column on matrix $a$.\\
%% Scope:{\bf local}.\\
%% Type:{\bf optional}\\
%% Specified as: an integer value
\end{description}

\begin{description}
\item[\bf On Return]
\item[a] the matrix into which elements will be inserted.\\
Scope:{\bf local}\\
Type:{\bf required}\\
Intent: {\bf inout}.\\
Specified as: a structured data of type \spdata.
\item[desc\_a] The communication descriptor.\\
Scope: {\bf local}. \\
Type: {\bf required}.\\
Intent: {\bf inout}.\\
Specified as: a variable of type \descdata.\\
\item[info] Error code.\\
Scope: {\bf local} \\
Type: {\bf required} \\
Intent: {\bf out}.\\
An integer value; 0 means no error has been detected. 
\end{description}

\section*{Notes}
\begin{enumerate}
\item On entry to this routine the descriptor may be in either the
  build or assembled state.
\item On entry to this routine the sparse matrix may be in either the
  build or update state. 
\item If the descriptor is in the build state, then the sparse matrix
  must also be in the build state; the action of the routine is to
  (implicitly) call \verb|psb_cdins| to add entries to the sparsity
  pattern; each sparse matrix entry implicitly defines a graph edge,
  that is passed to the descriptor routine for the appropriate
  processing. 
\item Any coefficients from matrix rows not assigned to the calling
  process are silently ignored;
\item If the descriptor is in the assembled state, then any entries in
  the sparse matrix that would generate additional communication
  requirements will be ignored; 
\item If the matrix is in the update state, any entries in positions
  that were not present in the original matrix will be ignored. 
\end{enumerate}

%
%%   psb_spasb %%
%
\subroutine{psb\_spasb}{Sparse matrix assembly routine}

\syntax{call psb\_spasb}{a, desc\_a, info, afmt, upd, dupl}

\begin{description}
\item[Type:] Synchronous.
\item[\bf On Entry]
\item[desc\_a] the communication descriptor.\\
Scope:{\bf local}.\\
Type:{\bf required}.\\
Intent: {\bf in}.\\
Specified as: a structured data of type \descdata.
\item[afmt] the storage format for the sparse matrix.\\
Scope: {\bf global}.\\
Type: {\bf optional}.\\
Intent: {\bf in}.\\
Specified as: an array of characters. Defalt:  'CSR'.
\item[upd] Provide for updates to the matrix coefficients.\\
Scope: {\bf global}.\\
Type: {\bf optional}.\\
Intent: {\bf in}.\\
Specified as: integer, possible values: \verb|psb_upd_srch_|, \verb|psb_upd_perm_|
\item[dupl] How to handle duplicate coefficients.\\
Scope: {\bf global}.\\
Type: {\bf optional}.\\
Intent: {\bf in}.\\
Specified as: integer, possible values: \verb|psb_dupl_ovwrt_|,
\verb|psb_dupl_add_|, \verb|psb_dupl_err_|.
\end{description}

\begin{description}
\item[\bf On Return]
\item[a] the matrix to be assembled.\\
Scope:{\bf local}\\
Type:{\bf required}\\
Intent: {\bf inout}.\\
Specified as: a structured data of type \spdata.
\item[info] Error code.\\
Scope: {\bf local} \\
Type: {\bf required} \\
Intent: {\bf out}.\\
An integer value; 0 means no error has been detected. 
\end{description}

\section*{Notes}
\begin{enumerate}
\item On entry to this routine the descriptor must  be in  the
  assembled state, i.e. \verb|psb_cdasb| must already have been called.
\item The sparse matrix may be in either the build or update state;
\item Duplicate entries are detected and handled in both build and
  update state, with the exception of the error action that is only
  taken in the build state, i.e. on the first assembly; 
\item If the update choice is \verb|psb_upd_perm_|, then subsequent
  calls to \verb|psb_spins| to update the matrix must be arranged in
  such a way as to produce exactly the same sequence of coefficient
  values as encountered at the first assembly; 
\item On exit from this routine the matrix is in the assembled state,
  and thus is suitable for the computational routines. 
\end{enumerate}



%% %
%% %%   psb_spcnv %%
%% %
%% \subroutine{psb\_spcnv}{Converts a sparse matrix storage format}

%% \syntax{call psb\_spcnv}{a, b, desc\_a, info}

%% \begin{description}
%% \item[\bf On Entry]
%% \item[a] the matrix to be converted.\\
%% Scope:{\bf local}\\
%% Type:{\bf required}\\
%% Specified as: a structured data of type \spdata.
%% \item[desc\_a] the communication descriptor.\\
%% Scope:{\bf local}.\\
%% Type:{\bf required}.\\
%% Specified as: a structured data of type \descdata.
%% \end{description}

%% \begin{description}
%% \item[\bf On Return]
%% \item[b] the converted matrix.\\
%% Scope:{\bf local}\\
%% Type:{\bf required}\\
%% Specified as: a structured data of type \spdata.
%% \item[info] Error code.
%% Scope: {\bf local} \\
%% Type: {\bf required}\\
%% Specified as: an integer variable.
%% \end{description}



%
%%   psb_spfree %%
%
\subroutine{psb\_spfree}{Frees a sparse matrix}

\syntax{call psb\_spfree}{a, desc\_a, info}

\begin{description}
\item[Type:] Synchronous.
\item[\bf On Entry]
\item[a] the matrix to be freed.\\
Scope:{\bf local}\\
Type:{\bf required}\\
Intent: {\bf inout}.\\
Specified as: a structured data of type \spdata.
\item[desc\_a] the communication descriptor.\\
Scope:{\bf local}.\\
Type:{\bf required}.\\
Intent: {\bf in}.\\
Specified as: a structured data of type \descdata.
\end{description}

\begin{description}
\item[\bf On Return]
\item[info] Error code.\\
Scope: {\bf local} \\
Type: {\bf required} \\
Intent: {\bf out}.\\
An integer value; 0 means no error has been detected. 
\end{description}




%
%%   psb_sprn %%
%
\subroutine{psb\_sprn}{Reinit sparse matrix structure for psblas routines.}

\syntax{call psb\_sprn}{a, decsc\_a, info, clear}

\begin{description}
\item[Type:] Synchronous.
\item[\bf On Entry]
\item[a] the matrix to be reinitialized.\\
Scope:{\bf local}\\
Type:{\bf required}\\
Intent: {\bf inout}.\\
Specified as: a structured data of type \spdata.
\item[desc\_a] the communication descriptor.\\
Scope:{\bf local}.\\
Type:{\bf required}.\\
Intent: {\bf in}.\\
Specified as: a structured data of type \descdata.
\item[clear] Choose whether to zero out matrix coefficients\\
Scope:{\bf local}.\\
Type:{\bf optional}.\\
Intent: {\bf in}.\\
Default: true.
\end{description}

\begin{description}
\item[\bf On Return]
\item[info] Error code.\\
Scope: {\bf local} \\
Type: {\bf required} \\
Intent: {\bf out}.\\
An integer value; 0 means no error has been detected. 
\end{description}
\section*{Notes}
\begin{enumerate}
\item On exit from this routine the sparse matrix is in the update
  state. 
\end{enumerate}
%
%%   psb_spupdate %%
%
%% \subroutine{psb\_spupdate}{Updates a sparse matrix.}

%% \syntax{call psb\_spupdate}{a, ia, ja, blck, desc\_a, info, ix, jx, updflag}

%% \begin{description}
%% \item[\bf On Entry]
%% \end{description}

%% \begin{description}
%% \item[\bf On Return]
%% \end{description}
%% %
%% %%   psb_csrp %%
%% %
%% \subroutine{psb\_csrp}{Applies a right permutation to a sparse matrix}

%% \syntax{call psb\_csrp}{trans, iperm, a, info}

%% \begin{description}
%% \item[\bf On Entry]
%% \item[trans] A character that specifies whether to permute $A$  or $A^T$.\\
%% Scope: {\bf local} \\
%% Type: {\bf required}\\
%% Specified as: a single character with value 'N' for $A$ or 'T' for $A^T$.\\
%% \item[iperm] An integer array containing permutation information.\\
%% Scope: {\bf local} \\
%% Type: {\bf required}\\
%% Specified as: an integer one-dimensional array.\\
%% \item[a] The sparse matrix to be permuted.\\
%% Scope: {\bf local} \\
%% Type: {\bf required}\\
%% Specified as: a \spdata variable.\\
%% \begin{description}
%% \item[\bf On Return]
%% \item[info] Error code.\\
%% Scope: {\bf local} \\
%% Type: {\bf required}\\
%% Specified as: Integer scalar.\\
%% \end{description}


%
%%   psb_alloc %%
%
\subroutine{psb\_geall}{Allocates a dense matrix}

\syntax{call psb\_geall}{x, desc\_a, info, n}

\begin{description}
\item[Type:] Synchronous.
\item[\bf On Entry]
\item[desc\_a] The communication descriptor.\\
Scope: {\bf local} \\
Type: {\bf required}\\
Intent: {\bf in}.\\
Specified as: a variable of type \descdata.\\
\item[n] The number of columns of the dense matrix to be allocated.\\
Scope: {\bf local} \\
Type: {\bf optional}\\
Intent: {\bf in}.\\
Specified as: Integer scalar, default $1$. It is ignored if $x$ is a
rank-1 array. 
\end{description}

\begin{description}
\item[\bf On Return]
\item[x] The dense matrix to be allocated.\\
Scope: {\bf local} \\
Type: {\bf required}\\
Intent: {\bf out}.\\
Specified as: a rank one or two array with the ALLOCATABLE
attribute, of type real, complex or integer.\\
\item[info] Error code.\\
Scope: {\bf local} \\
Type: {\bf required} \\
Intent: {\bf out}.\\
An integer value; 0 means no error has been detected. 
\end{description}


%
%%   psb_ins %%
%
\subroutine{psb\_geins}{Dense matrix insertion routine}

\syntax{call psb\_geins}{m, irw, val, x, desc\_a, info,dupl}

\begin{description}
\item[Type:] Asynchronous.
\item[\bf On Entry]
\item[m] Number of rows in $val$  to be inserted.\\
Scope:{\bf local}.\\
Type:{\bf required}.\\
Intent: {\bf in}.\\
Specified as: an integer value.
\item[irw] Indices of the rows to be inserted. Specifically, row $i$
  of $val$ will be inserted into the local row corresponding to the
  global row index $irw(i)$.
Scope:{\bf local}.\\
Type:{\bf required}.\\
Intent: {\bf in}.\\
Specified as: an integer array.
\item[val] the dense submatrix to be inserted.\\
Scope:{\bf local}.\\
Type:{\bf required}.\\
Intent: {\bf in}.\\
Specified as: a rank 1 or 2  array.
Specified as: an integer value.
\item[desc\_a] the communication descriptor.\\
Scope:{\bf local}.\\
Type:{\bf required}.\\
Intent: {\bf in}.\\
Specified as: a structured data of type \descdata.
\item[dupl] How to handle duplicate coefficients.\\
Scope: {\bf global}.\\
Type: {\bf optional}.\\
Intent: {\bf in}.\\
Specified as: integer, possible values: \verb|psb_dupl_ovwrt_|,
\verb|psb_dupl_add_|.
\end{description}

\begin{description}
\item[\bf On Return]
\item[x] the output dense matrix.\\
Scope: {\bf local} \\
Type: {\bf required}\\
Intent: {\bf inout}.\\
Specified as: a rank one or two array with the ALLOCATABLE
attribute, of type real, complex or integer.\\
\item[info] Error code.\\
Scope: {\bf local} \\
Type: {\bf required} \\
Intent: {\bf out}.\\
An integer value; 0 means no error has been detected. 
\end{description}

\section*{Notes}
\begin{enumerate}
\item Dense vectors/matrices do not have an associated state;
\item Duplicate entries are either overwritten or added, there is no
  provision for raising an error condition. 
\end{enumerate}


%
%%   psb_asb %%
%
\subroutine{psb\_geasb}{Assembly a dense matrix}

\syntax{call psb\_geasb}{x, desc\_a, info}

\begin{description}
\item[Type:] Synchronous.
\item[\bf On Entry]
\item[desc\_a] The communication descriptor.\\
Scope: {\bf local} \\
Type: {\bf required}\\
Intent: {\bf in}.\\
Specified as: a variable of type \descdata.\\
\end{description}

\begin{description}
\item[\bf On Return]
\item[x] The dense matrix to be assembled.\\
Scope: {\bf local} \\
Type: {\bf required}\\
Intent: {\bf inout}.\\
Specified as: a rank one or two array with the ALLOCATABLE
attribute, of type real, complex or integer.\\
\item[info] Error code.\\
Scope: {\bf local} \\
Type: {\bf required} \\
Intent: {\bf out}.\\
An integer value; 0 means no error has been detected. 
\end{description}
%
%%   psb_free %%
%
\subroutine{psb\_gefree}{Frees a dense matrix}

\syntax{call psb\_gefree}{x, desc\_a, info}

\begin{description}
\item[Type:] Synchronous.
\item[\bf On Entry]
\item[x] The dense matrix to
  be freed.\\
Scope: {\bf local} \\
Type: {\bf required}\\
Intent: {\bf inout}.\\
Specified as: a rank one or two array with the ALLOCATABLE
attribute, of type real, complex or integer.\\

\item[desc\_a] The communication descriptor.\\
Scope: {\bf local} \\
Type: {\bf required}\\
Intent: {\bf in}.\\
Specified as: a variable of type \descdata.\\
\end{description}

\begin{description}
\item[\bf On Return]
\item[info] Error code.\\
Scope: {\bf local} \\
Type: {\bf required} \\
Intent: {\bf out}.\\
An integer value; 0 means no error has been detected. 
\end{description}


%
%%   psb_gelp %%
%
\subroutine{psb\_gelp}{Applies a left permutation to a dense matrix}

\syntax{call psb\_gelp}{trans, iperm, x, info}

\begin{description}
\item[Type:] Asynchronous.
\item[\bf On Entry]
\item[trans] A character that specifies whether to permute $A$  or $A^T$.\\
Scope: {\bf local} \\
Type: {\bf required}\\
Intent: {\bf in}.\\
Specified as: a single character with value 'N' for $A$ or 'T' for $A^T$.\\
\item[iperm] An integer array containing permutation information.\\
Scope: {\bf local} \\
Type: {\bf required}\\
Intent: {\bf in}.\\
Specified as: an integer one-dimensional array.\\
\item[x] The dense matrix to be permuted.\\
Scope: {\bf local} \\
Type: {\bf required}\\
Intent: {\bf inout}.\\
Specified as: a one or two dimensional array.\\
\end{description}

\begin{description}
\item[\bf On Return]
\item[info] Error code.\\
Scope: {\bf local} \\
Type: {\bf required} \\
Intent: {\bf out}.\\
An integer value; 0 means no error has been detected. 
\end{description}


%
%%   psb_glob_to_loc %%
%
\subroutine{psb\_glob\_to\_loc}{Global to local indices convertion}

\syntax{call psb\_glob\_to\_loc}{x, y, desc\_a, info, iact,owned}
\syntax*{call psb\_glob\_to\_loc}{x, desc\_a, info, iact,owned}

\begin{description}
\item[Type:] Asynchronous.
\item[\bf On Entry]
\item[x] An integer vector of indices to be converted.\\
Scope: {\bf local} \\
Type: {\bf required}\\
Intent: {\bf in, inout}.\\
Specified as: a rank one integer array.\\
\item[desc\_a] the communication descriptor.\\
Scope:{\bf local}.\\
Type:{\bf required}.\\
Intent: {\bf in}.\\
Specified as: a structured data of type \descdata.
\item[iact] specifies action to be taken in case of range errors. 
Scope: {\bf global} \\
Type: {\bf optional}\\
Intent: {\bf in}.\\
Specified as: a character variable  \verb|I|gnore, \verb|W|arning or
\verb|A|bort, default \verb|I|gnore.
\item[owned] Specfies valid range of input 
Scope: {\bf global} \\
Type: {\bf optional}\\
Intent: {\bf in}.\\
If true, then only indices strictly owned by the current process are
considered valid, if false then halo indices are also
accepted. Default: false. 
\end{description}

\begin{description}
\item[\bf On Return]
\item[x] If $y$ is not present,
  then $x$ is overwritten with the translated integer indices. 
Scope: {\bf global} \\
Type: {\bf required}\\
Intent: {\bf inout}.\\
Specified as: a rank one integer array.
\item[y] If $y$ is  present,
  then $y$ is overwritten with the translated integer indices, and $x$
  is left unchanged. 
Scope: {\bf global} \\
Type: {\bf optional}\\
Intent: {\bf out}.\\
Specified as: a rank one integer array.
\item[info] Error code.\\
Scope: {\bf local} \\
Type: {\bf required} \\
Intent: {\bf out}.\\
An integer value; 0 means no error has been detected. 
\end{description}

\section*{Notes}
\begin{enumerate}
\item If an input index is out of range, then the corresponding output
  index is set to a negative number;
\item The default \verb|I|gnore means that the negative output is the
  only action taken on an out-of-range input.
\end{enumerate}


%
%%   psb_loc_to_glob %%
%
\subroutine{psb\_loc\_to\_glob}{Local to global indices conversion}

\syntax{call psb\_loc\_to\_glob}{x, y, desc\_a, info, iact}
\syntax*{call psb\_loc\_to\_glob}{x, desc\_a, info, iact}

\begin{description}
\item[Type:] Asynchronous.
\item[\bf On Entry]
\item[x] An integer vector of indices to be converted.\\
Scope: {\bf local} \\
Type: {\bf required}\\
Intent: {\bf in, inout}.\\
Specified as: a rank one integer array.\\
\item[desc\_a] the communication descriptor.\\
Scope:{\bf local}.\\
Type:{\bf required}.\\
Intent: {\bf in}.\\
Specified as: a structured data of type \descdata.
\item[iact] specifies action to be taken in case of range errors. 
Scope: {\bf global} \\
Type: {\bf optional}\\
Intent: {\bf in}.\\
Specified as: a character variable  \verb|I|gnore, \verb|W|arning or
\verb|A|bort, default \verb|I|gnore.
\end{description}

\begin{description}
\item[\bf On Return]
\item[x] If $y$ is not present,
  then $x$ is overwritten with the translated integer indices. 
Scope: {\bf global} \\
Type: {\bf required}\\
Intent: {\bf inout}.\\
Specified as: a rank one integer array.
\item[y] If $y$ is not present,
  then $y$ is overwritten with the translated integer indices, and $x$
  is left unchanged. 
Scope: {\bf global} \\
Type: {\bf optional}\\
Intent: {\bf out}.\\
Specified as: a rank one integer array.
\item[info] Error code.\\
Scope: {\bf local} \\
Type: {\bf required} \\
Intent: {\bf out}.\\
An integer value; 0 means no error has been detected. 
\end{description}




%
%%   psb_ins %%
%
\subroutine{psb\_get\_boundary}{Extract list of boundary elements}

\syntax{call psb\_get\_boundary}{bndel, desc, info}

\begin{description}
\item[Type:] Asynchronous.
\item[\bf On Entry]
\item[desc] the communication descriptor.\\
Scope:{\bf local}.\\
Type:{\bf required}.\\
Intent: {\bf in}.\\
Specified as: a structured data of type \descdata.
\end{description}

\begin{description}
\item[\bf On Return]
\item[bndel] The list of boundary elements on the calling process, in
  local numbering.\\
Scope: {\bf local} \\
Type: {\bf required}\\
Intent: {\bf out}.\\
Specified as: a rank one array with the ALLOCATABLE
attribute, of type integer.\\
\item[info] Error code.\\
Scope: {\bf local} \\
Type: {\bf required} \\
Intent: {\bf out}.\\
An integer value; 0 means no error has been detected. 
\end{description}

\section*{Notes}
\begin{enumerate}
\item If there are no boundary elements (i.e., if the local part of
  the connectivity graph is self-contained) the output vector is set
  to the ``not allocated'' state. 
\item Otherwise the size of \verb|bndel| will be exactly equal to the
  number of boundary elements. 
\end{enumerate}

\subroutine{psb\_get\_overlap}{Extract list of overlap elements}

\syntax{call psb\_get\_overlap}{ovrel, desc, info}

\begin{description}
\item[Type:] Asynchronous.
\item[\bf On Entry]
\item[desc] the communication descriptor.\\
Scope:{\bf local}.\\
Type:{\bf required}.\\
Intent: {\bf in}.\\
Specified as: a structured data of type \descdata.
\end{description}

\begin{description}
\item[\bf On Return]
\item[ovrel] The list of overlap elements on the calling process, in
  local numbering.\\
Scope: {\bf local} \\
Type: {\bf required}\\
Intent: {\bf out}.\\
Specified as: a rank one array with the ALLOCATABLE
attribute, of type integer.\\
\item[info] Error code.\\
Scope: {\bf local} \\
Type: {\bf required} \\
Intent: {\bf out}.\\
An integer value; 0 means no error has been detected. 
\end{description}

\section*{Notes}
\begin{enumerate}
\item If there are no overlap elements the output vector is set
  to the ``not allocated'' state.  
\item Otherwise the size of \verb|ovrel| will be exactly equal to the
  number of overlap elements. 
\end{enumerate}



\subroutine{psb\_sp\_getrow}{Extract row(s) from a sparse matrix}

\syntax{call psb\_sp\_getrow}{row, a, nz, ia, ja, val, info, append,
  nzin, lrw}

\begin{description}
\item[Type:] Asynchronous.
\item[\bf On Entry]
\item[row] The (first) row to be extracted.\\
Scope:{\bf local}\\
Type:{\bf required}\\
Intent: {\bf in}.\\
Specified as: an integer $>0$.
\item[a] the matrix from  which to get rows.\\
Scope:{\bf local}\\
Type:{\bf required}\\
Intent: {\bf in}.\\
Specified as: a structured data of type \spdata.
\item[append] Whether to append or overwrite existing output.\\
Scope:{\bf local}\\
Type:{\bf optional}\\
Intent: {\bf in}.\\
Specified as: a logical value default: false (overwrite).
\item[nzin] Input size to be appended to.\\
Scope:{\bf local}\\
Type:{\bf optional}\\
Intent: {\bf in}.\\
Specified as: an integer $>0$. When append is true, specifies how many
entries in the output vectors are already filled. 
\item[lrw] The last  row to be extracted.\\
Scope:{\bf local}\\
Type:{\bf optional}\\
Intent: {\bf in}.\\
Specified as: an integer $>0$, default: $row$.

%% \item[is] the starting row on matrix $a$.\\
%% Scope:{\bf local}.\\
%% Type:{\bf optional}.\\
%% Specified as: an integer vaule.
%% \item[js] the starting column on matrix $a$.\\
%% Scope:{\bf local}.\\
%% Type:{\bf optional}\\
%% Specified as: an integer value
\end{description}

\begin{description}
\item[\bf On Return]
\item[nz] the number of elements returned by this call.\\
Scope:{\bf local}.\\
Type:{\bf required}.\\
Intent: {\bf out}.\\
Returned  as: an integer scalar.
\item[ia] the row indices.\\
Scope:{\bf local}.\\
Type:{\bf required}.\\
Intent: {\bf inout}.\\
Specified as: an integer array with the \verb|ALLOCATABLE| attribute.
\item[ja] the column indices of the elements to be inserted.\\
Scope:{\bf local}.\\
Type:{\bf required}.\\
Intent: {\bf inout}.\\
Specified as: an integer array with the \verb|ALLOCATABLE| attribute.
\item[val] the elements to be inserted.\\
Scope:{\bf local}.\\
Type:{\bf required}.\\
Intent: {\bf inout}.\\
Specified as: a real  array with the \verb|ALLOCATABLE| attribute.
\item[info] Error code.\\
Scope: {\bf local} \\
Type: {\bf required} \\
Intent: {\bf out}.\\
An integer value; 0 means no error has been detected. 
\end{description}

\section*{Notes}
\begin{enumerate}
\item The output $nz$ is always the size of the output generated by
  the current call; thus, if \verb|append=.true.|, the total output
  size will be $nzin+nz$, with the newly extracted coefficients stored in
  entries \verb|nzin+1:nzin+nz| of the array arguments;
\item When \verb|append=.true.|  the output arrays are reallocated as
  necessary;
\item The row and column indices are returned in the local numbering
  scheme; if the global numbering is desired, the user may employ the
  \verb|psb_loc_to_glob| routine on the output.
\end{enumerate}





\subroutine{psb\_sizeof}{Memory occupation}

This function computes the memory occupation of a PSBLAS object.


\syntax{psb\_sizeof}{a}
\syntax*{psb\_sizeof}{desc\_a}
\syntax*{psb\_sizeof}{prec}

\begin{description}
\item[Type:] Asynchronous.
\item[\bf On Entry]
\item[a] A sparse matrix
$A$. \\ 
Scope: {\bf local} \\
Type: {\bf required}\\
Intent: {\bf in}.\\
Specified as: a structured data of type \spdata.
\item[desc\_a] Communication descriptor.\\
Scope: {\bf local} \\
Type: {\bf required}\\
Intent: {\bf in}.\\
Specified as: a structured data of type \descdata.
\item[prec] 
Scope: {\bf local} \\
Type: {\bf required}\\
Intent: {\bf in}.\\
Specified as: a preconditioner data structure \precdata.
\item[\bf On Return] 
\item[Function value] The memory occupation of the object specified in
  the calling sequence, in bytes.\\
Scope: {\bf local} \\
Returned  as: an integer number.
\end{description}



%%% Local Variables: 
%%% mode: latex
%%% TeX-master: "userguide"
%%% End: 
